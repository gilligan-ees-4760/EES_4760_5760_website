% Options for packages loaded elsewhere
\PassOptionsToPackage{unicode}{hyperref}
\PassOptionsToPackage{hyphens}{url}
%
\documentclass[
]{article}
\usepackage{amsmath,amssymb}
\usepackage{iftex}
\ifPDFTeX
  \usepackage[T1]{fontenc}
  \usepackage[utf8]{inputenc}
  \usepackage{textcomp} % provide euro and other symbols
\else % if luatex or xetex
  \usepackage{unicode-math} % this also loads fontspec
  \defaultfontfeatures{Scale=MatchLowercase}
  \defaultfontfeatures[\rmfamily]{Ligatures=TeX,Scale=1}
\fi
\usepackage{lmodern}
\ifPDFTeX\else
  % xetex/luatex font selection
\fi
% Use upquote if available, for straight quotes in verbatim environments
\IfFileExists{upquote.sty}{\usepackage{upquote}}{}
\IfFileExists{microtype.sty}{% use microtype if available
  \usepackage[]{microtype}
  \UseMicrotypeSet[protrusion]{basicmath} % disable protrusion for tt fonts
}{}
\makeatletter
\@ifundefined{KOMAClassName}{% if non-KOMA class
  \IfFileExists{parskip.sty}{%
    \usepackage{parskip}
  }{% else
    \setlength{\parindent}{0pt}
    \setlength{\parskip}{6pt plus 2pt minus 1pt}}
}{% if KOMA class
  \KOMAoptions{parskip=half}}
\makeatother
\usepackage{xcolor}
\usepackage[margin=1in]{geometry}
\usepackage{graphicx}
\makeatletter
\def\maxwidth{\ifdim\Gin@nat@width>\linewidth\linewidth\else\Gin@nat@width\fi}
\def\maxheight{\ifdim\Gin@nat@height>\textheight\textheight\else\Gin@nat@height\fi}
\makeatother
% Scale images if necessary, so that they will not overflow the page
% margins by default, and it is still possible to overwrite the defaults
% using explicit options in \includegraphics[width, height, ...]{}
\setkeys{Gin}{width=\maxwidth,height=\maxheight,keepaspectratio}
% Set default figure placement to htbp
\makeatletter
\def\fps@figure{htbp}
\makeatother
\setlength{\emergencystretch}{3em} % prevent overfull lines
\providecommand{\tightlist}{%
  \setlength{\itemsep}{0pt}\setlength{\parskip}{0pt}}
\setcounter{secnumdepth}{-\maxdimen} % remove section numbering
\usepackage{jglucida}
\usepackage{float}
\usepackage{booktabs}
\usepackage[version=4]{mhchem}
\usepackage{siunitx}

\newcommand{\Ntwo}{\ce{N2}}
\newcommand{\Otwo}{\ce{O2}}
\newcommand{\Htwo}{\ce{H2}}
\newcommand{\COO}{\ce{CO2}}
\newcommand{\methane}{\ce{CH4}}

\newcommand{\water}{\ce{H2O}}
\newcommand{\silica}{\ce{SiO2}}
\newcommand{\calcite}{\ce{CaCO3}}
\newcommand{\CaSi}{\ce{CaSiO3}}
\newcommand{\carbonic}{\ce{H2CO3}}
\newcommand{\carbonicacid}{\carbonic}

\newcommand{\Hplus}{\ce{H+}}
\newcommand{\OH}{\ce{OH-}}
\newcommand{\Caplus}{\ce{Ca^2+}}
\newcommand{\Siplus}{\ce{Si^2+}}
\newcommand{\carb}{\ce{CO3^2-}}
\newcommand{\bicarb}{\ce{HCO3-}}
\newcommand{\bicarbonate}{\bicarb}
\newcommand{\carbonate}{\carb}
\newcommand{\silicate}{\ce{SiO3^2-}}

\newcommand{\degC}{\ensuremath{^\circ \mathrm{C}}}
\newcommand{\degF}{\ensuremath{^\circ \mathrm{F}}}
\newcommand{\pH}{p\ce{H}}
\newcommand{\permil}{\permille}

\newcommand{\Osixteen}{\ce{^{16}O}}
\newcommand{\Oeightteen}{\ce{^{18}O}}
\newcommand{\Ctwelve}{\ce{^{12}C}}
\newcommand{\Cthirteen}{\ce{^{13}C}}
\newcommand{\Cfourteen}{\ce{^{14}C}}
\newcommand{\Hone}{\ce{^1H}}
\newcommand{\deuterium}{\ce{^2H}}
\newcommand{\tritium}{\ce{^3H}}

\newcommand{\SLUGULATOR}{\textsc{slugulator}}
\newcommand{\GEOCARB}{\textsc{geocarb}}
\newcommand{\MODTRAN}{\textsc{modtran}}
\ifLuaTeX
  \usepackage{selnolig}  % disable illegal ligatures
\fi
\usepackage{bookmark}
\IfFileExists{xurl.sty}{\usepackage{xurl}}{} % add URL line breaks if available
\urlstyle{same}
\hypersetup{
  pdftitle={Reading Assignments},
  pdfauthor={EES 4760/5760: Agent- and Individual-Based Computational Modeling},
  hidelinks,
  pdfcreator={LaTeX via pandoc}}

\title{Reading Assignments}
\author{EES 4760/5760: Agent- and Individual-Based Computational
Modeling}
\date{Fall, 2024}

\begin{document}
\maketitle

\section{General instructions for reading
assignments}\label{general-instructions-for-reading-assignments}

\begin{itemize}
\item
  Do the assigned reading \emph{before\/} you come to class on the date
  for which it is assigned. If you have questions or find the ideas
  presented in the readings confusing, I encourage you to ask questions
  in class.
\item
  Check the list of errata at
  \url{https://www.railsback-grimm-abm-book.com/downloads-errata-2nd-edition/}.
\item
  Questions in the ``Reading Notes'\,' sections of the assignments are
  for you to think about to make sure you understand the material, but
  you do not have to write up your answers or turn them in.

  You are responsible for all the assigned readings, but topics I have
  highlighted in the reading notes are particularly important.
\item
  In addition to the questions I ask in the reading notes, look over the
  ``Conclusions'' section at the end of each chapter in
  \emph{Agent-Based Modeling} to check whether you understand the key
  facts and concepts from the chapter.
\end{itemize}

\subsection{Disclaimer}\label{disclaimer}

This is a schedule of reading assignments through the entire term. I
have worked hard to plan the semester, but I may need to deviate from
this schedule, if I decide that it's important to spend more time on
some subjects.

The most up-to-date versions of the reading assignments will be posted
on the ``Schedule'' page of the
\href{https://ees4760.jgilligan.org/schedule}{course web site}:
\url{https://ees4760.jgilligan.org/schedule}

\section{Wed., Aug.~21: Introduction}\label{wed.-aug.-21-introduction}

\subsection{Reading:}\label{reading}

No new reading for today.

\section{Mon., Aug.~26: The computer modeling
cycle}\label{mon.-aug.-26-the-computer-modeling-cycle}

\subsection{Reading:}\label{reading-1}

\subsubsection{Required Reading
(everyone):}\label{required-reading-everyone}

\begin{itemize}
\tightlist
\item
  Agent-Based and Individual-Based Modeling, Ch. 1.
\item
  Handout: \href{/files/reading/Tyson_Artificial_Societies_1997.pdf}{P.
  Tyson, ``Artificial Societies,'' Technology Review \textbf{100} (3),
  15--17 (1997).}.
\end{itemize}

\subsubsection{Optional Extra Reading:}\label{optional-extra-reading}

\begin{itemize}
\tightlist
\item
  Modeling Social Behavior, Ch. 1.
\end{itemize}

\subsubsection{Reading Notes:}\label{reading-notes}

This reading sets the stage for answering the questions:

\begin{enumerate}
\def\labelenumi{\arabic{enumi}.}
\tightlist
\item
  What is computational modeling and why is it useful in social and
  natural science research?
\item
  What are agent based models? How are they different from other kinds
  of models? What makes them useful for scientific research?
\end{enumerate}

The reading introduces the idea of a \textbf{modeling cycle}. You should
understand the different steps in the modeling cycle. You should also
think about why Railsback and Grimm describe modeling as a cycle, as
opposed to a linear process with a start and stop.

As to what makes agent-based modeling special, Steven Railsback and
Volker Grimm are ecologists and \emph{Agent-Based Modeling} emphasizes
aspects of agent-based modeling that are well suited for studying
ecological systems. Others, such as social scientists, emphasize the
aspects of agent-based modeling that are well suited for problems in
social science. And still others, such as computer scientists, emphasize
aspects of automated and autonomous things (ranging from packets of data
on a network to swarms of robots or flying drones that need to
coordinate their activities and avoid collisions). What all of these
approaches have in common are their use of individuals or
\textbf{agents} (what is an agent?), which inhabit some kind of space or
\textbf{environment} (this could be physical space or an abstract space,
such as a computer network). Agents \textbf{interact} with each other
and with the environment, and they make \textbf{decisions} according to
rules.

The article ``Artificial Life'' gives you a feel for how an early
agent-based model called ``Sugarscape'' was used as part of a very
influential research project in the 1990s. Joshua Epstein and Robert
Axtell who wrote Sugarscape are highly respected pioneers in agent-based
modeling and the Sugarscape model set off a revolution in agent-based
modeling by showing that a very simple model could reproduce complex
phenomena that are observed in real societies. As you read through this
article, think about what the different applications of agent-based
models have in common. Do these suggest questions that you might be
interested in exploring with agent-based models. Do you have questions,
as you read this, about whether computer modeling can really tell you
about real societies?

The chapter from \emph{Modeling Social Behavior} is optional, but
strongly recommended for graduate students and for students interested
in applications of agent-based modeling to social systems and social
science.

Agent-based models are often used to examine \textbf{emergent}
phenomena. Neither reading describes clearly what \emph{emergence}
means. There is no simple definition, but during the semester we will
pay a lot of attention to learning about emergence and trying to
understand it. Do not worry if you don't understand emergence at this
point. Emergence is difficult to put into words, and it's much easier to
understand from experience. Over the course of the semester, we will
work together to understand what emergence is and how to study it.

\section{Wed., Aug.~28: Introduction to
NetLogo}\label{wed.-aug.-28-introduction-to-netlogo}

\subsection{Reading:}\label{reading-2}

\subsubsection{Required Reading
(everyone):}\label{required-reading-everyone-1}

\begin{itemize}
\tightlist
\item
  Agent-Based and Individual-Based Modeling, Ch. 2.
\end{itemize}

\subsubsection{Optional Extra Reading:}\label{optional-extra-reading-1}

\begin{itemize}
\tightlist
\item
  Modeling Social Behavior, Ch. 2.
\end{itemize}

\subsubsection{Reading Notes:}\label{reading-notes-1}

Familiarize yourself with NetLogo. I recommend that you read through the
chapter with NetLogo open on your computer. Feel free to play around
with NetLogo and try things out. The homework consists of following the
step-by-step creation of a model in section 2.3.

The chapter from \emph{Modeling Social Behavior} is purely optional. You
don't need to read it, and everything you need to know is covered in the
reading from Railsback \& Grimm, but it may be interesting to read this
alternate introduction to the NetLogo modeling system and programming
language.

\section{Mon., Sep.~2: Specifying models: The ODD
protocol}\label{mon.-sep.-2-specifying-models-the-odd-protocol}

\subsection{Reading:}\label{reading-3}

\subsubsection{Required Reading
(everyone):}\label{required-reading-everyone-2}

\begin{itemize}
\tightlist
\item
  Agent-Based and Individual-Based Modeling, Ch. 3.
\end{itemize}

\subsubsection{Reading Notes:}\label{reading-notes-2}

Read carefully through the different design principles. Some of them
have meanings that are a bit different from what you might infer from
colloquial use.

For instance:

\begin{itemize}
\tightlist
\item
  \textbf{Adaptation} does not mean a persistent change in a turtle's
  behavior similar to the biological/Darwinian sense of adaptation in
  species. Rather, it means the way an agent changes its behavior in
  response to its \emph{immediate} conditions. Thus, adaptation in the
  ODD sense might include behaviors such as eating when you are hungry
  (\emph{eating} is an \textbf{adaptation} to \emph{hunger}), putting on
  warmer clothing when it's cold out (bundling up is an adaptation to
  cold), and running away from a predator.
\item
  The kind of persistent changes that arise over time from experience
  fall under the ODD design concept of \textbf{learning}: If there is
  more food near a river than on hills, turtles may \textbf{learn} to go
  to rivers when they are hungry.
\end{itemize}

You can download several useful documents related to the ODD protocol
from the class web site:

\begin{itemize}
\tightlist
\item
  The journal article, \href{/files/odd/Grimm_2010_ODD_update.pdf}{V.
  Grimm \emph{et al.} (2010). ``The ODD protocol: A review and first
  update'' \emph{Ecological Modeling} \textbf{221}, 2760--68.}.
  \url{https://ees4760.jgilligan.org/files/odd/Grimm_2010_ODD_update.pdf}
\item
  A Word document that provides
  \href{/files/odd/Grimm_2010_odd_template.docx}{a template for writing
  ODDs}:
  \url{https://ees4760.jgilligan.org/files/odd/Grimm_2010_odd_template.docx}
\item
  Lists of scientific publications using agent-based and
  individual-based models that either do or don't use the ODD protocol
  (this appeared as \href{/files/odd/Grimm_2010_appendix_1.pdf}{Appendix
  1} of the Grimm \emph{et al.} paper):

  \begin{itemize}
  \tightlist
  \item
    \url{https://ees4760.jgilligan.org/files/odd/Grimm_2010_appendix_1.pdf},
  \item
    \url{https://ees4760.jgilligan.org/files/odd/ch3_ex1_pubs_with_no_ODD.pdf},
  \item
    \url{https://ees4760.jgilligan.org/files/odd/ch3_ex2_pubs_with_ODD.pdf}.
  \end{itemize}
\end{itemize}

\section{Wed., Sep.~4: Your first
model}\label{wed.-sep.-4-your-first-model}

\subsection{Reading:}\label{reading-4}

\subsubsection{Required Reading
(everyone):}\label{required-reading-everyone-3}

\begin{itemize}
\tightlist
\item
  Agent-Based and Individual-Based Modeling, Ch. 4.
\end{itemize}

\subsubsection{Reading Notes:}\label{reading-notes-3}

For the reading, read Chapter 4 in \emph{Agent-Based Modeling} first and
focus mostly on this chapter.

It is worth noting that the Sugarscape model you read about for Aug.~27
is very similar to the hilltopping model. Sugarscape was part of a very
influential research project in the 1990s, in which Joshua Epstein and
Robert Axtell showed that a very simple model could reproduce complex
biological and economic phenomena that are observed in real societies
and ecosystems. You may find it interesting at this point to look back
at the ``Artificial Societies'' article and to play with the Sugarscape
model in NetLogo.

You can download an
\href{/files/models/chapter_04/ButterflyModelODD.txt}{ODD for the
butterfly model}, which is suitable to paste into the ``Info'' tab in
NetLogo, from the class web site:
\url{https://ees4760.jgilligan.org/files/models/chapter_04/ButterflyModelODD.txt}

\section{Mon., Sep.~9: Using models for
science}\label{mon.-sep.-9-using-models-for-science}

\subsection{Reading:}\label{reading-5}

\subsubsection{Required Reading
(everyone):}\label{required-reading-everyone-4}

\begin{itemize}
\tightlist
\item
  Agent-Based and Individual-Based Modeling, Ch. 5.
\end{itemize}

\subsubsection{Optional Extra Reading:}\label{optional-extra-reading-2}

\begin{itemize}
\tightlist
\item
  Modeling Social Behavior, Ch. 3.
\end{itemize}

\subsubsection{Reading Notes:}\label{reading-notes-4}

This reading sets the stage for answering the big question, ``How can we
use agent-based models to do science?'' There are several aspects to
this question, which this chapter will introduce:

\begin{enumerate}
\def\labelenumi{\arabic{enumi}.}
\tightlist
\item
  How can we produce quantitative output from our models?
\item
  How can your models read and write data to and from files? (This is
  important for connecting your model to other parts of your project)
\item
  How should we test our models to make sure they do what we think they
  do? (More on this in Chapter 6)
\item
  Making your research reproducible by using version control and
  documentation.
\end{enumerate}

A number of you may like to use Excel or statistical analysis tools,
such as \texttt{R}, \texttt{SPSS}, or \texttt{Stata}. The material in
this chapter about importing and exporting data using text or
\texttt{.csv} files will be very useful for this. By default, NetLogo
only allows you to read in data in simple text files. However, if comes
with some extensions that you can use to read in data from other common
file formats, including \texttt{.csv} and ArcGIS shapefiles and raster
(grid) files.

If you want to read in data from csv files, you may want to look at the
documentation for the \texttt{csv} extension to NetLogo. To use it, you
just put the line \texttt{includes\ {[}csv{]}} as the first line of your
model, and then use functions from the extension, such as
\texttt{let\ data\ csv:from-file\ "myfile.csv"}.

To read in date from ArcGIS files, look at the documentation for the
\texttt{GIS} extension. You would put the line
\texttt{extensions\ {[}\ gis\ {]}} as the first line of your model, and
then use functions, such as \texttt{gis:load-dataset}, which can load
vector shape files (\texttt{.shp}) and raster grid files (\texttt{.grd}
or \texttt{.asc}). The GIS extension offers a lot of functions for
working with vector and raster GIS data. If you're interested in using
GIS data in your models, take a look at the GIS examples in the NetLogo
model library.

You can download the
\href{/files/models/chapter_05/ElevationData.txt}{data file} with the
elevations for the realistic butterfly model from
\url{https://ees4760.jgilligan.org/files/models/chapter_05/ElevationData.txt}

The chapter from \emph{Modeling Social Behavior} is an optional
supplementary reading. I strongly recommend that graduate students and
students interested in social systems and social science read this.

The main reading for today, from Railsback \& Grimm presents a model of
an ecological system. This chapter presents a famous agent-based model
of racial segregation in housing. This model is historically important,
and also controversial. It was perhaps the first agent-based model ever
used to study a research problem in social sicence, and it was written
by a researcher who went on to win the Nobel Prize in economics.
However, the model is problematic and has been criticized because it is
often interpreted in ways that minimize the role of institutional racism
in driving segregation (e.g., government policies that explicitly
prohibited racially integrated housing across large parts of the entire
United States). This chapter presents the model and also discusses the
challenges of using it effectively for science and the importance of
considering the actual historical context of the social system being
modeled. Specifically, this model does not account for the historical
segregationist policies, so it's important not to assume that
experiments using this model can tell us about real-world segregation in
the U.S., but nonetheless, the model can help us identicy potentially
important obstacles to remedying the segregated housing patterns that
those policies produced.

\section{Wed., Sep.~11: Testing and validating
models}\label{wed.-sep.-11-testing-and-validating-models}

\subsection{Reading:}\label{reading-6}

\subsubsection{Required Reading
(everyone):}\label{required-reading-everyone-5}

\begin{itemize}
\tightlist
\item
  Agent-Based and Individual-Based Modeling, Ch. 6.
\end{itemize}

\subsubsection{Reading Notes:}\label{reading-notes-5}

No one writes perfect programs. Errors in programs controlling medical
equipment have killed people. Errors in computer models and data
analysis code have not had such dire results, but have wasted lots of
time for researchers and have caused public policy to proceed on
incorrect assumptions. In many cases, these errors were uncovered only
after a great deal of frustration because the original researchers would
not share their computer codes with others who were suspicious of their
results.

You can never be certain that your model is correct, but the more
aggressively you check for errors the more confident you can be that it
does not have major problems.

Two very important things you can do to ensure that your research does
not suffer a similar fate are:

\begin{enumerate}
\def\labelenumi{\arabic{enumi}.}
\item
  Test your code. Assume your program has errors in it and make the
  search for those errors a priority in your programming process. Some
  things you can do in this regard are:

  \begin{itemize}
  \tightlist
  \item
    Write your code with tests that will help you find errors.
  \item
    Work with a partner: after one of you writes code, the other should
    read it and check for errors.
  \item
    Break your program up into small chunks. It is easier to test and
    find bugs if you are looking at a short block of code than if you
    are looking at hundreds of lines of code.
  \item
    Independently reimplement submodels and check whether they agree
    with the submodel you are using.
  \end{itemize}
\item
  Publish your code. If you trust your results and believe they are
  important enough to publish in a book or journal, then you should make
  your code available (there are many free sites, such as
  \href{https://github.com}{\texttt{github.com}} and
  \href{https://openabm.org}{\texttt{openabm.org}} where people can
  publish their models and other computer code).

  The more that other researchers can read your code, the greater the
  probability that they will find any errors, and if you make it easy
  for others to use your code, it will help science because other people
  can build on your work, and it will help your reputation because when
  other people use your model or other code they are likely to cite the
  publication in which you first announced it, so your work will get
  attention.

  Many scholarly journals demand that you make your code available as a
  condition for publishing your paper, and federal funding agencies are
  increasingly requiring that any research funded by their grants must
  make its code and data available to other researchers and the public.
\end{enumerate}

\textbf{The Cultural Dissemination Model}

The
\href{//files/models/chapter_06/axelrod_culture_dissemination_1997.pdf}{paper
describing the culture dissemination model} and a
\href{/files/models/chapter_06/CultureDissemination_Untested.nlogo}{NetLogo
model} that implements the ODD, but with many errors, can be downloaded
from the class web site:

\begin{itemize}
\tightlist
\item
  \url{https://ees4760.jgilligan.org/files/models/chapter_06/axelrod_culture_dissemination_1997.pdf},
\item
  and
  \url{https://ees4760.jgilligan.org/files/models/chapter_06/CultureDissemination_Untested.nlogo}.
\end{itemize}

\section{Mon., Sep.~16: Choosing Research
Projects}\label{mon.-sep.-16-choosing-research-projects}

\subsection{Reading:}\label{reading-7}

\subsubsection{Required Reading
(everyone):}\label{required-reading-everyone-6}

\begin{itemize}
\tightlist
\item
  Agent-Based and Individual-Based Modeling, Ch. 7.
\end{itemize}

\subsubsection{Reading Notes:}\label{reading-notes-6}

There is not very much reading for today. Read Chapter 7 of
\emph{Agent-Based Modeling} carefully (it's very short) The point of
this chapter is to help you get ideas for your term research project.

Before class, I want you to read through the research project assignment
and think about what you might want to do for a research project. We
will spend class talking about possible research projects.

\section{Wed., Sep.~18: Emergence}\label{wed.-sep.-18-emergence}

\subsection{Reading:}\label{reading-8}

\subsubsection{Required Reading
(everyone):}\label{required-reading-everyone-7}

\begin{itemize}
\tightlist
\item
  Agent-Based and Individual-Based Modeling, Ch. 8.
\end{itemize}

\subsubsection{Reading Notes:}\label{reading-notes-7}

This is a major chapter. Emergence is one of the most important concepts
in agent-based modeling, so pay close attention to the discussion in
this chapter and think about how you can measure and assess emergence.

This chapter also introduces a very important tool for doing experiments
in NetLogo: \textbf{BehaviorSpace}. BehaviorSpace lets you repeatedly
run a NetLogo model while varying the settings of any of the controls on
your user interface. Where there is randomness (stochasticity) in the
model, you can perform many runs at each set of control settings. This
will let us perform \textbf{sensitivity analysis} to determine whether a
certain emergent phenomenon we are investigating happens only for values
of the parameters within a narrow range, or whether it happens over a
wide range of the parameters. It will let us determine which parameters
are most important for the phenomenon.

For homework and your modeling projects you will use BehaviorSpace
extensively. BehaviorSpace outputs large amounts of data to
\texttt{.csv} files, which you can read into Excel, R, SPSS, or another
tool where you can do statistical analysis and generate plots such as
the ones in figures 8.3, 8.5, and 8.6.

The format in which BehaviorSpace saves its data is very annoying to
deal with in many tools. Indeed, it's almost impossible to do anything
useful with it in Excel. Because of this, I have written a tool called
\texttt{analyzeBehaviorspace} that can read the output of a
BehaviorSpace run and allow you to interactively graph it and
re-organize the data to make it more useful.

You can either use this tool online in a web browser at
\url{https://analyze-behaviorspace.jgilligan.org} or install it on your
own computer. For details, see the description of
\texttt{analyzeBehaviorspace} on the ``Reading Resources and Computing
Tools for Research'' handout.

As you read the chapter, be sure to try out the experiments with the
birth-and-death model and the flocking model. Try reading the output of
the behaviorspace runs into analyzeBehaviorspace (the web version or a
local version installed on your computer), or your favorite statistical
analysis software and try to generate plots similar to figures 8.3, 8.5,
and 8.6.

If you have time, try to play around with BehaviorSpace using those
models (varying different parameters) or other models from the NetLogo
library to explore the ways that changing parameters affects the models'
behavior.

\section{Mon., Sep.~23: Observation}\label{mon.-sep.-23-observation}

\subsection{Reading:}\label{reading-9}

\subsubsection{Required Reading
(everyone):}\label{required-reading-everyone-8}

\begin{itemize}
\tightlist
\item
  Agent-Based and Individual-Based Modeling, Ch. 9.
\end{itemize}

\subsubsection{Reading Notes:}\label{reading-notes-8}

In this chapter, we examine how to detect and record the properties of a
model that we want to study.

The article, D. Kornhauser,
\href{https://jasss.soc.surrey.ac.uk/12/2/1.html}{U. Wilensky, and W.
Rand. (2009). ``Design guidelines for agent-based model visualization,''
\emph{Journal of Artificial Societies and Social Simulation}
\textbf{12}, 1} is available online at
\url{https://jasss.soc.surrey.ac.uk/12/2/1.html}.

I have posted a
\href{/files/models/chapter_09/ch9_ex8_netlogo_exercises.pdf}{refresher
guide} for NetLogo programming on the class web site at
\url{https://ees4760.jgilligan.org/files/models/chapter_09/ch9_ex8_netlogo_exercises.pdf}

\section{Wed., Sep.~25: Sensing}\label{wed.-sep.-25-sensing}

\subsection{Reading:}\label{reading-10}

\subsubsection{Required Reading
(everyone):}\label{required-reading-everyone-9}

\begin{itemize}
\tightlist
\item
  Agent-Based and Individual-Based Modeling, Ch. 10.
\end{itemize}

\subsubsection{Reading Notes:}\label{reading-notes-9}

In addition to the assigned reading from the textbook, read the ODD for
your team project before you come to class. The ODDs are posted on
Brightspace and \href{/projects/}{the class web site}.

For some of the class period, you and your team will start translating
the ODD for your project model into NetLogo code.

Important programming concepts in this chapter include:

\begin{description}
\tightlist
\item[Links:]
Agents interact with their physical environment (patches around them)
and with other agents nearby, but they can also interact in social
networks, which can be represented by links.
\item[Variable scope:]
Understand the differences between global variables, local variables,
patch variables, agent variables, and link variables. Understand how an
agent can get the value of a global variable or variables belonging to a
certain patch or link or another agent.
\item[Entity detection:]
Understand different ways to detect which agents or patches meet certain
conditions (e.g., within a certain distance, have a certain color, have
the largest or smallest values of some variable, etc.).
\end{description}

The agents' interactions, both with their environment and with each
other through sensing. Part of the design concepts section of a model's
ODD consists of specifying what the agents can sense: They might be able
to sense other agents within a certain distance. They might only be able
to detect other agents if they are within a certain angle (e.g., the
agent might be able to look forward, but might not be able to see behind
itself unless it turns around). Agents might be able to detect certain
qualities of one another (e.g., I can see how tall you are, but I can't
see how much money you have).

Agents can interact both spatially and through networks of links. You
can create many kinds of links so that agents can belong to many
networks (e.g., family, co-workers, members of a church congregation,
etc.).

Sensing involves two steps:

\begin{enumerate}
\def\labelenumi{\arabic{enumi}.}
\tightlist
\item
  Detect which entities your agent (or patch) will sense.
\item
  Get the values of the sensed variables from those entities.
\end{enumerate}

Be sure to code the Business Investor model as you read section 10.4.
You will also use it in Chapters 11 and 12, and it will form the basis
for one of the team projects.

\section{Mon., Sep.~30: Adaptive Behavior and
Objectives}\label{mon.-sep.-30-adaptive-behavior-and-objectives}

\subsection{Reading:}\label{reading-11}

\subsubsection{Required Reading
(everyone):}\label{required-reading-everyone-10}

\begin{itemize}
\tightlist
\item
  Agent-Based and Individual-Based Modeling, Ch. 11.
\end{itemize}

\subsubsection{Reading Notes:}\label{reading-notes-10}

Agents' behavior often consists of trying to achieve some
\textbf{objective}.

I have discussed the way that Adam Smith's ``invisible hand of the
market'' is a kind of agent-based view of a nation's economy: Each
person (agent) has an objective of trying to maximize his or her own
wealth (that's the agent's \textbf{micromotive}), and in doing so, the
population of agents manages unintentionally to maximize the total
wealth of the nation (an emergent \textbf{macrobehavior} that results
from the collective interactions of the agents and their micromotives).

For Darwin, agents whose objectives are to survive and reproduce under
changing environmental conditions achieve emergent phenomena of
evolution and speciation.

If we are going to program an agent-based model to simulate such an
economy (we saw this in the Sugarscape models), you need to program your
agents to try to achieve their objective (maximize their wealth). There
are two approaches to this:

\begin{enumerate}
\def\labelenumi{\arabic{enumi}.}
\tightlist
\item
  You could program a sophisticated strategy into your agents.
\item
  You could program a simple strategy into your agents, but give them
  the ability to learn from their experience and adapt their behavior
  according to what they learn. (see section 11.3 for details and an
  example)
\end{enumerate}

This chapter discusses different kinds of objectives you might have your
agents employ. An important concept from decision theory and behavioral
economics that might be new to you is \textbf{satisficing}. This term,
introduced by Herbert Simon\footnote{Herbert Simon (1916--2001) was a
  fascinating intellectual. Kind of a renaissance scholar, he made major
  contributions to political science, economics, cognitive psychology,
  and artificial intelligence. He won the Nobel Prize for economics in
  1978. His publications have been cited more than 250,000 times and
  even \texttt{r\ lubridate::year(lubridate::today())\ -\ 2001} years
  after his death, they are still cited more than 10,000 times per year.}
in 1956, refers to making decisions by choosing a ``good-enough'' option
when it would take too much time and effort to determine which option is
the absolute best. See section 11.4 for details and an example.

\section{Wed., Oct.~2: Prediction}\label{wed.-oct.-2-prediction}

\subsection{Reading:}\label{reading-12}

\subsubsection{Required Reading
(everyone):}\label{required-reading-everyone-11}

\begin{itemize}
\tightlist
\item
  Agent-Based and Individual-Based Modeling, Ch. 12.
\end{itemize}

\subsubsection{Reading Notes:}\label{reading-notes-11}

In class, we will discuss the telemarketer model.

For the teams working on the telemarketer model, the steps are:

\begin{enumerate}
\def\labelenumi{\arabic{enumi}.}
\tightlist
\item
  Build the model as described in the ODD.
\item
  Do the analyses in section 13.3.2
\item
  Next work on two extensions:

  \begin{enumerate}
  \def\labelenumii{\arabic{enumii}.}
  \tightlist
  \item
    Mergers (section 13.5)
  \item
    Customers remember (section 13.6)
  \end{enumerate}
\end{enumerate}

Before class, everyone should read Chapter 13 and the ODD for the
telemarketer model and be ready to discuss the model in class. This will
be a chance for the teams working on the model to ask questions.

\section{Mon., Oct.~7: Interaction}\label{mon.-oct.-7-interaction}

\subsection{Reading:}\label{reading-13}

\subsubsection{Required Reading
(everyone):}\label{required-reading-everyone-12}

\begin{itemize}
\tightlist
\item
  Agent-Based and Individual-Based Modeling, Ch. 13.
\end{itemize}

\subsubsection{Optional Extra Reading:}\label{optional-extra-reading-3}

\begin{itemize}
\tightlist
\item
  Modeling Social Behavior, Ch. 6--7, 9.
\end{itemize}

\subsubsection{Reading Notes:}\label{reading-notes-12}

The optional reading from \emph{Modeling Social Behavior} may be
interesting. Chapters 6--7 apply agent-based modeling to game-theory and
chapter 9 looks into social networks among agents in detail.

Chapter 6 focuses on a class of strategic games, such as the Prisoner's
Dilemma, in which the players do better when they cooperate with each
other, but the game gives each player has a strong tempation to betray
their partner. The chapter investigates a rich line of research, which
shows that there may be reasons for natural selection to favor
cooperation, so people may have evolved to be mostly trustworthy and
cooperative.

Chapoter 7 focuses on coordination: The Prisoner's Dilemma game in
chapter 6 looked at people who interact with a small number of
neighbors, whom they can get to know well from experience. This chapter
looks at coordination among large numbers of people, where there is no
opportunity to develop that kind of deep knowledge of your neighbors'
character and trustworthiness.

Chapter 9 presents more detail about social networks and the use of
NetLogo links. The main reading from Railsback \& Grimm covers the
basics of links, but this chapter discusses the different kinds of
social networks (different topologies) that can form using those links.

\section{Wed., Oct.~9: Team
Presentations}\label{wed.-oct.-9-team-presentations}

\subsection{Reading:}\label{reading-14}

No new reading for today.

\section{Mon., Oct.~14: Fall Break}\label{mon.-oct.-14-fall-break}

Fall Break, no class.

\section{Wed., Oct.~16: Research Project
ODDs}\label{wed.-oct.-16-research-project-odds}

\subsection{Reading:}\label{reading-15}

No new reading for today.

\section{Mon., Oct.~21: Scheduling}\label{mon.-oct.-21-scheduling}

\subsection{Reading:}\label{reading-16}

\subsubsection{Required Reading
(everyone):}\label{required-reading-everyone-13}

\begin{itemize}
\tightlist
\item
  Agent-Based and Individual-Based Modeling, Ch. 14.
\end{itemize}

\subsubsection{Reading Notes:}\label{reading-notes-13}

A good example of asynchronous updating is the
\href{/files/models/chapter_23/Ch_23_4_breeding_synchrony.nlogo}{model
of breeding synchrony}, described in
\href{/files/models/chapter_05/Jovani_Grimm_2008_Breeding.pdf}{Jovani
and Grimm (2008)} and in Chapter 23, which I have posted on the class
web site:
\url{https://ees4760.jgilligan.org/files/models/chapter_23/Ch_23_4_breeding_synchrony.nlogo}

\section{Wed., Oct.~23: Stochasticity}\label{wed.-oct.-23-stochasticity}

\subsection{Reading:}\label{reading-17}

\subsubsection{Required Reading
(everyone):}\label{required-reading-everyone-14}

\begin{itemize}
\tightlist
\item
  Agent-Based and Individual-Based Modeling, Ch. 15.
\end{itemize}

\section{Mon., Oct.~28: Collectives}\label{mon.-oct.-28-collectives}

\subsection{Reading:}\label{reading-18}

\subsubsection{Required Reading
(everyone):}\label{required-reading-everyone-15}

\begin{itemize}
\tightlist
\item
  Agent-Based and Individual-Based Modeling, Ch. 16.
\end{itemize}

\section{Wed., Oct.~30: Patterns}\label{wed.-oct.-30-patterns}

\subsection{Reading:}\label{reading-19}

\subsubsection{Required Reading
(everyone):}\label{required-reading-everyone-16}

\begin{itemize}
\tightlist
\item
  Agent-Based and Individual-Based Modeling, Ch. 17--18.
\end{itemize}

\subsubsection{Reading Notes:}\label{reading-notes-14}

You can download \href{/files/models/chapter_18/ch18_before_ODD.pdf}{the
ODD for the BEFORE beech forest model}, which is described in section
18.3, from the class web site:
\url{https://ees4760.jgilligan.org/files/models/chapter_18/ch18_before_ODD.pdf}.

I have also posted
\href{/files/models/chapter_18/ch18_ex1_models_list.pdf}{a list of
published models} in which observed patterns are important:
\url{https://ees4760.jgilligan.org/files/models/chapter_18/ch18_ex1_models_list.pdf}

\section{Mon., Nov.~4: Theory
Development}\label{mon.-nov.-4-theory-development}

\subsection{Reading:}\label{reading-20}

\subsubsection{Required Reading
(everyone):}\label{required-reading-everyone-17}

\begin{itemize}
\tightlist
\item
  Agent-Based and Individual-Based Modeling, Ch. 19.
\end{itemize}

\subsubsection{Optional Extra Reading:}\label{optional-extra-reading-4}

\begin{itemize}
\tightlist
\item
  Modeling Social Behavior, Ch. 8.
\end{itemize}

\subsubsection{Reading Notes:}\label{reading-notes-15}

There is an implementation of the
\href{/files/models/chapter_19/ch19_ex2_wood_hoopoes.nlogo}{wood hoopoe
model}, suitable for Exercise 2, on the class web site:
\url{https://ees4760.jgilligan.org/files/models/chapter_19/ch19_ex2_wood_hoopoes.nlogo}

The optional chapter from \emph{Modeling Social Behavior} discusses
other approaches to using agent-based models to do science. The readings
from Railsback \& Grimm focus on the methodology of Pattern-Oriented
Modeling. This chapter looks at other ways of thinking about connecting
agent-basedm models to scientific methods. The focus here is on
statistical analysis of experimental results and the use of Bayesian
statistical methods to identify and avoid pitfalls where experiemnts may
fool researchers into drawing the wrong conclusion (e.g., thinking a
hypothesis is true, when it's really false).

\section{Wed., Nov.~6: Parameterization and
Calibration}\label{wed.-nov.-6-parameterization-and-calibration}

\subsection{Reading:}\label{reading-21}

\subsubsection{Required Reading
(everyone):}\label{required-reading-everyone-18}

\begin{itemize}
\tightlist
\item
  Agent-Based and Individual-Based Modeling, Ch. 20.
\end{itemize}

\section{Mon., Nov.~11: Parameterization and Calibration
2}\label{mon.-nov.-11-parameterization-and-calibration-2}

\subsection{Reading:}\label{reading-22}

\subsubsection{Required Reading
(everyone):}\label{required-reading-everyone-19}

\begin{itemize}
\tightlist
\item
  Agent-Based and Individual-Based Modeling, Ch. 20.
\end{itemize}

\section{Wed., Nov.~13: Analyzing
ABMs}\label{wed.-nov.-13-analyzing-abms}

\subsection{Reading:}\label{reading-23}

\subsubsection{Required Reading
(everyone):}\label{required-reading-everyone-20}

\begin{itemize}
\tightlist
\item
  Agent-Based and Individual-Based Modeling, Ch. 22.
\end{itemize}

\section{Mon., Nov.~18: Sensitivity and
Robustness}\label{mon.-nov.-18-sensitivity-and-robustness}

\subsection{Reading:}\label{reading-24}

\subsubsection{Required Reading
(everyone):}\label{required-reading-everyone-21}

\begin{itemize}
\tightlist
\item
  Agent-Based and Individual-Based Modeling, Ch. 23.
\end{itemize}

\subsubsection{Reading Notes:}\label{reading-notes-16}

You can download the
\href{/files/models/chapter_23/Ch_23_4_breeding_synchrony.nlogo}{model
of breeding synchrony}, described in
\href{/files/models/chapter_05/Jovani_Grimm_2008_Breeding.pdf}{Jovani
and Grimm (2008)} and in section 23.4, from the class web site:
\url{https://ees4760.jgilligan.org/files/models/chapter_23/Ch_23_4_breeding_synchrony.nlogo}

\section{Wed., Nov.~20: Looking Ahead: ABMs Beyond this
Course}\label{wed.-nov.-20-looking-ahead-abms-beyond-this-course}

\subsection{Reading:}\label{reading-25}

\subsubsection{Required Reading
(everyone):}\label{required-reading-everyone-22}

\begin{itemize}
\tightlist
\item
  Agent-Based and Individual-Based Modeling, Ch. 24.
\end{itemize}

\subsubsection{Optional Extra Reading:}\label{optional-extra-reading-5}

\begin{itemize}
\tightlist
\item
  Modeling Social Behavior, Ch. 10--11.
\end{itemize}

\subsubsection{Reading Notes:}\label{reading-notes-17}

The optional reading from \emph{Modeling Social Behavior} provides
perspectives on the relationship between simulation models and the real
world and how to use models wisely by appreciating both what they can do
and what they can't do, what parts of the real world they capture
faithfully and which parts they don't.

\section{Mon., Nov.~25--Wed., Nov.~27: Thanksgiving
Break}\label{mon.-nov.-25wed.-nov.-27-thanksgiving-break}

Thanksgiving Break, no class.

\section{Mon., Dec.~2: Presentations}\label{mon.-dec.-2-presentations}

\subsection{Reading:}\label{reading-26}

No new reading for today.

\section{Wed., Dec.~4: Presentations}\label{wed.-dec.-4-presentations}

\subsection{Reading:}\label{reading-27}

No new reading for today.

\end{document}
