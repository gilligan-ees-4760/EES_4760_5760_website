\section{Meet Your Professor}
Jonathan Gilligan has worked in many areas of science and public policy.
His past research includes work on laser physics, quantum optics,
laser surgery, electrical properties of the heart, using modified spy planes to
study the ozone layer in the stratosphere, and connections between religion and
care for the environment.
\iffalse

Professor Gilligan joined the Vanderbilt Faculty in 1994 as a member of the
Department of Physics and Astronomy. In 2003, when the Department of Geology
became the Department of Earth and Environmental Science, Professor Gilligan
joined the new department to focus on atmospheric science, global climate change,
and the interactions of politics, ethics, religion, communication, and the
environment.
\fi

Professor Gilligan is a member of the Vanderbilt Institute for Energy and Environment,
a founding member of the Vanderbilt Initiative for Smart-City Operations Research (VISOR),
a founding member of the Erd\H{o}s institute for Collaboration, Innovation, and
Entrepreneurship,
and the Associate Director for Research at the Vanderbilt Climate Change Research Network.

His current research investigates the role of individual and household behavior
in greenhouse gas emissions in the United States;
how ``smart cities'' can use technology to reduce environmental footprints and
promote health and citizen empowerment;
water conservation policies in American cities;
vulnerability and resilience to environmental stress in Bangladesh;
adaptation to water scarcity in Sri Lanka;
and developing new directions for climate policy in the US.

In 2017, Professor Gilligan and Professor Michael Vandenbergh shared
the Morrison Prize for the highest-impact paper on sustainability law and
policy published in the previous year. Professors Gilligan and Vandenbergh
have developed this work into a book, \emph{Beyond Politics: The Private
  Governance Approach to Climate Change\/} (Cambridge University Press, 2017).

Apart from his academic work, Professor Gilligan dabbles in writing for the
theater. His stage adaptation of Nathaniel Hawthorne's \emph{The Scarlet Letter},
co-written with his mother Carol Gilligan, has been staged at The Culture
Project in New York City, starring
Marisa Tomei, Ron Cephas Jones, and Bobby Cannavale, and was later performed
at Prime Stage Theatre, Pittsburgh and in a touring production by The National
Players.
Prof.\ Gilligan and Carol Gilligan also wrote the libretto for an opera,
\emph{Pearl}, in collaboration composer Amy Scurria, and producer/conductor
Sara Jobin, which was performed at Shakespeare \& Company in Lenox MA,
starring Maureen O'Flynn, John Bellemer, Marnie Breckenridge, John Cheek,
and Michael Corvino, and in Shanghai China,
% as part of a cultural exchange,
starring Li Xin, Wang Yang, John Bellemer, and Lin Shu.