% Options for packages loaded elsewhere
\PassOptionsToPackage{unicode}{hyperref}
\PassOptionsToPackage{hyphens}{url}
%
\documentclass[
]{article}
\usepackage{amsmath,amssymb}
\usepackage{iftex}
\ifPDFTeX
  \usepackage[T1]{fontenc}
  \usepackage[utf8]{inputenc}
  \usepackage{textcomp} % provide euro and other symbols
\else % if luatex or xetex
  \usepackage{unicode-math} % this also loads fontspec
  \defaultfontfeatures{Scale=MatchLowercase}
  \defaultfontfeatures[\rmfamily]{Ligatures=TeX,Scale=1}
\fi
\usepackage{lmodern}
\ifPDFTeX\else
  % xetex/luatex font selection
\fi
% Use upquote if available, for straight quotes in verbatim environments
\IfFileExists{upquote.sty}{\usepackage{upquote}}{}
\IfFileExists{microtype.sty}{% use microtype if available
  \usepackage[]{microtype}
  \UseMicrotypeSet[protrusion]{basicmath} % disable protrusion for tt fonts
}{}
\makeatletter
\@ifundefined{KOMAClassName}{% if non-KOMA class
  \IfFileExists{parskip.sty}{%
    \usepackage{parskip}
  }{% else
    \setlength{\parindent}{0pt}
    \setlength{\parskip}{6pt plus 2pt minus 1pt}}
}{% if KOMA class
  \KOMAoptions{parskip=half}}
\makeatother
\usepackage{xcolor}
\usepackage[margin=1in]{geometry}
\usepackage{longtable,booktabs,array}
\usepackage{calc} % for calculating minipage widths
% Correct order of tables after \paragraph or \subparagraph
\usepackage{etoolbox}
\makeatletter
\patchcmd\longtable{\par}{\if@noskipsec\mbox{}\fi\par}{}{}
\makeatother
% Allow footnotes in longtable head/foot
\IfFileExists{footnotehyper.sty}{\usepackage{footnotehyper}}{\usepackage{footnote}}
\makesavenoteenv{longtable}
\usepackage{graphicx}
\makeatletter
\newsavebox\pandoc@box
\newcommand*\pandocbounded[1]{% scales image to fit in text height/width
  \sbox\pandoc@box{#1}%
  \Gscale@div\@tempa{\textheight}{\dimexpr\ht\pandoc@box+\dp\pandoc@box\relax}%
  \Gscale@div\@tempb{\linewidth}{\wd\pandoc@box}%
  \ifdim\@tempb\p@<\@tempa\p@\let\@tempa\@tempb\fi% select the smaller of both
  \ifdim\@tempa\p@<\p@\scalebox{\@tempa}{\usebox\pandoc@box}%
  \else\usebox{\pandoc@box}%
  \fi%
}
% Set default figure placement to htbp
\def\fps@figure{htbp}
\makeatother
\setlength{\emergencystretch}{3em} % prevent overfull lines
\providecommand{\tightlist}{%
  \setlength{\itemsep}{0pt}\setlength{\parskip}{0pt}}
\setcounter{secnumdepth}{-\maxdimen} % remove section numbering
\usepackage{jglucida}
\usepackage{float}
\usepackage{booktabs}
\usepackage[version=4]{mhchem}
\usepackage{siunitx}

\newcommand{\Ntwo}{\ce{N2}}
\newcommand{\Otwo}{\ce{O2}}
\newcommand{\Htwo}{\ce{H2}}
\newcommand{\COO}{\ce{CO2}}
\newcommand{\methane}{\ce{CH4}}

\newcommand{\water}{\ce{H2O}}
\newcommand{\silica}{\ce{SiO2}}
\newcommand{\calcite}{\ce{CaCO3}}
\newcommand{\CaSi}{\ce{CaSiO3}}
\newcommand{\carbonic}{\ce{H2CO3}}
\newcommand{\carbonicacid}{\carbonic}

\newcommand{\Hplus}{\ce{H+}}
\newcommand{\OH}{\ce{OH-}}
\newcommand{\Caplus}{\ce{Ca^2+}}
\newcommand{\Siplus}{\ce{Si^2+}}
\newcommand{\carb}{\ce{CO3^2-}}
\newcommand{\bicarb}{\ce{HCO3-}}
\newcommand{\bicarbonate}{\bicarb}
\newcommand{\carbonate}{\carb}
\newcommand{\silicate}{\ce{SiO3^2-}}

\newcommand{\degC}{\ensuremath{^\circ \mathrm{C}}}
\newcommand{\degF}{\ensuremath{^\circ \mathrm{F}}}
\newcommand{\pH}{p\ce{H}}
\newcommand{\permil}{\permille}

\newcommand{\Osixteen}{\ce{^{16}O}}
\newcommand{\Oeightteen}{\ce{^{18}O}}
\newcommand{\Ctwelve}{\ce{^{12}C}}
\newcommand{\Cthirteen}{\ce{^{13}C}}
\newcommand{\Cfourteen}{\ce{^{14}C}}
\newcommand{\Hone}{\ce{^1H}}
\newcommand{\deuterium}{\ce{^2H}}
\newcommand{\tritium}{\ce{^3H}}

\newcommand{\SLUGULATOR}{\textsc{slugulator}}
\newcommand{\GEOCARB}{\textsc{geocarb}}
\newcommand{\MODTRAN}{\textsc{modtran}}
\usepackage{bookmark}
\IfFileExists{xurl.sty}{\usepackage{xurl}}{} % add URL line breaks if available
\urlstyle{same}
\hypersetup{
  pdftitle={Homework Assignments},
  pdfauthor={EES 4760/5760: Agent- and Individual-Based Computational Modeling},
  hidelinks,
  pdfcreator={LaTeX via pandoc}}

\title{Homework Assignments}
\author{EES 4760/5760: Agent- and Individual-Based Computational
Modeling}
\date{Fall, 2025}

\begin{document}
\maketitle

\section{General instructions for homework
assignments}\label{general-instructions-for-homework-assignments}

\begin{itemize}
\item
  Homework must be turned in on Brightspace by 11:59 pm on the due date
  unless the assignment gives a different time.

  \begin{itemize}
  \tightlist
  \item
    Late homework will lose 5\% for every day it is late, but will not
    lose more than 50\%, no matter how late it is (you can always get up
    to half credit for completing late homework).
  \end{itemize}
\item
  When a homework assignment has more than one file, I encourage you to
  make a Zip file containing all the different files for the assignment,
  and upload that to Brightspace.

  There are many free software apps for creating and working with Zip
  files, such as \href{https://www.7-zip.org/}{7Zip} (Windows) or
  \href{https://www.keka.io/en/}{Keka} or
  \href{https://www.izip.com/}{iZip} (Mac)

  \begin{itemize}
  \tightlist
  \item
    If a homework assignment has multiple exercises and each exercise
    has multiple files, I encourage you to make separate folders for
    each exercise (e.g., \texttt{ex\_4.1} or \texttt{ex\_4\_1} for
    exercise 4.1) and then put all of the folders into a Zip file and
    upload the Zip file to Brightspace. Then when I unzip it, the files
    for each exercise will be in a folder for that exercise.
  \end{itemize}
\item
  When a homework assignment calls for you to write a NetLogo model and
  answer questions about that model, you have three options:

  \begin{enumerate}
  \def\labelenumi{\arabic{enumi}.}
  \item
    Answer the question in the NetLogo model by editing the INFO page
    for the model and adding lines like:

\begin{verbatim}
# Exercise 2.3

When I ran the model, this happened ...
\end{verbatim}
  \item
    Answer the questions in a separate document (Word, text, PDF, etc.),
    so you might have a file \texttt{ex\_2\_3.nlogo} with the model and
    a file \texttt{ex\_2\_3.docx} with the answer to the question about
    that model, or you can use one document called something like
    \texttt{homework\_4.docx} that has the written answers to all the
    exercises:

    \begin{quote}
    \begin{itemize}
    \item
      \textbf{Exercise 2.3:} I ran the model and this happened\ldots{}
    \item
      \textbf{Exercise 3.3:} Here is how I would use a model to answer
      questions about the spread of COVID-19 \ldots{}
    \end{itemize}
    \end{quote}
  \item
    If the answer involves figures (diagrams, etc.) that are hard to
    make in Word, you might write the answer on paper, photograph it,
    and upload the photo as something like \texttt{ex\_2\_3.jpg}.
  \end{enumerate}
\end{itemize}

I encourage you to discuss homework assignments with your classmates.
Some assignments will explicitly tell you to work in a team with a
classmate. Even when the assignments do not specify working in teams, it
is still fine to work together on homework assignments, but
\textbf{unless the assignment tells you to work in a team, you must
actually do all the work yourself}. This means that you can ask a
classmate to explain how they solved a problem but you have to go
through the steps independently and put the answer in your own words,
not simply copy someone else's work.

\textbf{It is a violation of the honor code to turn in homework that
someone else has done for you or which you copied from someone else.} If
you are unsure about how the honor code applies to an assignment, please
ask me.

\subsection{Homework Policy: Extra-Credit
Options}\label{homework-policy-extra-credit-options}

Some homework problems are assigned to graduate students only.
Undergraduate students may choose to do any of these problems that
interest them, and will receive for extra credit if they do.

I assign grades based on the required work, so your grade will not be
affected by whether or not other students do extra credit problems (in
other words, I never curve grades downward).

If you did poorly on some homework problems, or did not turn in some
homework problems, you can make up some of that deficit by doing
extra-credit problems on subsequent assignments. Extra credit on
homework only counts toward your homework grade for the semester and
cannot compensate for missed work on other assignments, such as the team
project or research project.

\subsection{Disclaimer}\label{disclaimer}

This is a schedule of homework assignments through the entire term. I
have worked hard to plan the semester, but I may need to deviate from
this schedule.

The most up-to-date versions of the homework assignments will be posted
on the ``Schedule'' page of the
\href{https://ees4760.jgilligan.org/schedule}{course web site}:
\url{https://ees4760.jgilligan.org/schedule}

\section{Mon., Aug.~25: Set up
NetLogo}\label{mon.-aug.-25-set-up-netlogo}

\subsection{Homework}\label{homework}

\subsubsection{Preliminary Information}\label{preliminary-information}

\textbf{There is nothing for you to turn in}, but do the following task
to prepare for next week:

\subsubsection{Homework Exercises:}\label{homework-exercises}

\begin{itemize}
\tightlist
\item
  Download NetLogo version 6.4.0 from
  \url{https://ccl.northwestern.edu/netlogo/} and install it on your
  computer.
\end{itemize}

\section{Wed., Aug.~27: Introducing
NetLogo}\label{wed.-aug.-27-introducing-netlogo}

\subsection{Homework}\label{homework-1}

\subsubsection{Homework Exercises:}\label{homework-exercises-1}

\begin{itemize}
\item
  As you read along with Section 2.3, follow along on your computer and
  build the \emph{Mushroom Hunt} model by typing in the code shown in
  the textbook. The whole program is shown on pages 27--29. \textbf{Save
  your model as \texttt{mushroom\_hunt.nlogo}}.

  This exercise may seem very simple, but it is the first step toward
  learning how to program NetLogo and it will be an important first step
  toward writing your own models.

  After you are done typing in your model, try running it.

  \textbf{When you are done, turn your model in on Brightspace.}
\end{itemize}

\subsubsection{Notes on Homework:}\label{notes-on-homework}

I recommend getting together with a classmate and working together on
this assignment. Since the assignment consists of typing code in and
running it, do not worry about your work being identical to your
partner's. However, I strongly recommend that you type everything in
yourself because you will not learn if you just copy someone else's code
or download the code from a source on the web.

If you run into trouble and cannot make your model work, do not worry.
Ask a classmate for help, or email me (and attach your \texttt{.nlogo}
model file), or simply come to class with questions about the problems
you had getting your model to work.

\section{Mon., Sep.~1: Becoming familiar with
NetLogo}\label{mon.-sep.-1-becoming-familiar-with-netlogo}

\subsection{Homework}\label{homework-2}

\subsubsection{Preliminary Information}\label{preliminary-information-1}

This homework consists of reading and working through tutorials, so
there is nothing to turn in.

\subsubsection{Homework Exercises:}\label{homework-exercises-2}

\begin{itemize}
\tightlist
\item
  Everyone should do exercises 2.1 and 2.2 in Chapter 2 of Railsback \&
  Grimm. This consists of reading and working through tutorials, so
  there is nothing to turn in.
\end{itemize}

\section{Wed., Sep.~3: Experimenting with
NetLogo}\label{wed.-sep.-3-experimenting-with-netlogo}

\subsection{Homework}\label{homework-3}

\subsubsection{Preliminary Information}\label{preliminary-information-2}

Upload your work to Brightspace when you're done (Word or text files for
the descriptions and ODD document, and \texttt{.nlogo} files for your
NetLogo models).

When there is more than one file for a homework assignment, I recommend
that you make a Zip file containing all of the files for the different
parts of the assignment and upload the Zip file to Brighspace instead of
uploading the files separately. When a homework assignment has more than
one problem, and each problem has multiple files, it can be helpful to
have a separate folder for each problem, and put all of these folders
into the ZIP file that you turn in.

There are many free software apps for creating and working with Zip
files, such as \href{https://www.7-zip.org/}{7Zip} for Windows and
\href{https://www.keka.io/en/}{Keka} and
\href{https://www.izip.com}{iZip} for Mac.

Some problems in this homework set call for you to write a NetLogo model
and also answer questions about the model. For answering the questions,
you may take any of these approaches:

\begin{enumerate}
\def\labelenumi{\arabic{enumi}.}
\item
  Edit the INFO page for your NetLogo model and add the following (this
  example is for exercise 2.3)

\begin{verbatim}
# Exercise 2.3

When I ran the model ...
\end{verbatim}
\item
  Write your answer in a Word file or other document file (text, PDF,
  etc.) and put the homework answers there. You can either make a
  separate document for each exercise (e.g., answers for exercise 2.3 go
  in \texttt{ex\_2\_3.docx}, answers for exercise 3.3 go in
  \texttt{ex\_3\_3.docx}), or you can make a single document file (e.g.,
  with the title \texttt{homework\_4\_answers.docx}) and put the
  homework answers in there, as follows:

  \begin{quote}
  \begin{itemize}
  \tightlist
  \item
    \textbf{Exercise 2.3:} When I ran the model \ldots{}
  \item
    \textbf{Exercise 3.3:} This is how I would use a model to answer
    questions about the spread of COVID-19 \ldots{}
  \end{itemize}
  \end{quote}
\item
  If your answers include diagrams or other kinds of figures or
  illustrations that are a pain to do in Word, you can hand-write your
  answers (\textbf{legibly!}), take a photo of them, and upload the
  photo.
\end{enumerate}

\subsubsection{Homework Exercises:}\label{homework-exercises-3}

\begin{itemize}
\item
  Railsback \& Grimm, Chapter 2, exercises Exercises 2.3--2.4.

  For exercise 4 in chapter 2, you will make seven sequential
  modifications of the basic mushroom hunt model. Each step modifies the
  previous one, so the last model will have all the modifications from
  the bulleted list in Ex. 2.4. Save each model with a new name, such as
  \texttt{ex\_2\_4a.nlogo}, \texttt{ex\_2\_4b.nlogo}, \dots,
  \texttt{ex\_2\_4g.nlogo}
\item
  Railsback \& Grimm, Chapter 3 exercise 3.3.

  Write your answers in any convenient text format (a simple text file,
  a Word document, a \texttt{.pdf} file, or whatever suits you). Call
  the file \texttt{ex\_3\_3.docx} (or \texttt{ex\_3\_3.pdf}, etc.).
\end{itemize}

\subsubsection{Notes on Homework:}\label{notes-on-homework-1}

My advice for Chapter 3, Exercise 3 is don't be too ambitious with your
model, but keep it very simple. Don't worry about getting everything
right. If there are things you don't feel sure about or don't know how
to express, you should just write a parenthetical note in your ODD
document commenting on your difficulty. Come to class prepared to talk
about how this exercise went and where you felt confused about trying to
specify your model.

\section{Wed., Sep.~10: Science with models: Butterfly
mating}\label{wed.-sep.-10-science-with-models-butterfly-mating}

\subsection{Homework}\label{homework-4}

\subsubsection{Preliminary Information}\label{preliminary-information-3}

Some problems in this homework set call for you to write a NetLogo model
and also answer questions about the model. For answering the questions,
you may take any of these approaches:

\begin{enumerate}
\def\labelenumi{\arabic{enumi}.}
\item
  Edit the INFO page for your NetLogo model and add the following (this
  example is for exercise 5.4)

\begin{verbatim}
# Exercise 5.4

When I ran the model ...
\end{verbatim}
\item
  Write your answer in a Word file or other document file (text, PDF,
  etc.) with the title \texttt{homework\_answers.docx} and put the
  homework answers in there, as follows:

  \textbf{Exercise 5.4:} When I ran the model \ldots{}
\item
  If your answers include diagrams or other kinds of figures or
  illustrations that are a pain to do in Word, you can hand-write your
  answers (\textbf{legibly!}), take a photo of them, and upload the
  photo.
\end{enumerate}

\subsubsection{Homework Exercises:}\label{homework-exercises-4}

\begin{itemize}
\item
  \textbf{Everyone:}

  \begin{itemize}
  \item
    Railsback \& Grimm, Ch. 4, Ex. 4.2, 4.4

    For exercise 4.4, \textbf{instead of the way the exercise is
    described in the book, do the following:}

    \begin{itemize}
    \tightlist
    \item
      Try adding ``noise'' to the landscape by adding a random number to
      the patch elevation.

      \begin{enumerate}
      \def\labelenumi{\arabic{enumi}.}
      \item
        Add a switch to the interface and call it ``noise''
      \item
        In the \texttt{ask\ patches} statement in \texttt{to\ setup},
        change the elevation to this:

\begin{verbatim}
ask patches
[
  set elevation 200 + (100 * (sin (pxcor * 3.8) +
                              sin (pycor * 3.8)))
  if noise [
    set elevation elevation +
                  random-float 20.0 - 10.0
  ]
  set pcolor scale-color green elevation 0 400
]
\end{verbatim}
      \item
        In the \texttt{crt} statementin \texttt{to\ setup} (where you
        create the turtles), instead of
        \texttt{setxy\ random-pxcor\ random-pycor} write
        \texttt{setxy\ 71\ 71} to start the turtles in the middle of the
        hills. Write up answers to the following questions and turn them
        in on Brightspace:
      \item
        Compare the turtle behavior with \texttt{noise} off to
        \texttt{noise} on for several values of \texttt{q}. How does the
        noise affect the movement?
      \item
        Does this give you any insight into why the paths in the
        original (noise-free) model look artificial and unlike what you
        might expect butterflies to do in the real world?
      \end{enumerate}
    \end{itemize}
  \item
    Railsback \& Grimm, Ch. 5, Ex. 5.1, 5.2, 5.4, and 5.7.
  \end{itemize}
\item
  \textbf{Graduate Students,} also do the following:

  \begin{itemize}
  \tightlist
  \item
    Railsback \& Grimm, Ch. 5, exercise 5.8
  \end{itemize}
\end{itemize}

\subsubsection{Notes on Homework:}\label{notes-on-homework-2}

\begin{itemize}
\item
  \textbf{Everyone:} When there is more than one file for a homework
  assignment, I recommend that you make a Zip file containing all of the
  files for the different parts of the assignment and upload the Zip
  file to Brighspace instead of uploading the files separately. When a
  homework assignment has more than one problem, and each problem has
  multiple files, it can be helpful to have a separate folder for each
  problem, and put all of these folders into the ZIP file that you turn
  in.

  There are many free software apps for creating and working with Zip
  files, such as \href{https://www.7-zip.org/}{7Zip} for Windows and
  \href{https://www.keka.io/en/}{Keka} and
  \href{https://www.izip.com}{iZip} for Mac.

  For exercise 5.1, you should have three versions of the model. Each
  version adds new changes on top of the previous version:

  \begin{itemize}
  \tightlist
  \item
    one version of the model that incorporates all the changes (listed
    with triangular bullets in the book) in section 5.2 (pp.~64--68),
  \item
    one version that starts with the previous version from 5.2 and also
    incorporates the additional changes in section 5.4 (p.~70),
  \item
    one version that starts with the previous version from 5.4 and also
    incorporates the additional changes in section 5.5 (p.~73)
  \end{itemize}

  For exercise 5.2, look in the NetLogo dictionary for a command that
  does what you want. The point of this exercise is to start getting you
  used to looking for new NetLogo commands when you want to do something
  you haven't yet learned about.

  Exercises 5.4 and 5.7 ask you to answer a question about the modified
  model. Be sure to turn in an answer to the question:

  Here are three possibilities:

  \begin{enumerate}
  \def\labelenumi{\arabic{enumi}.}
  \item
    Edit the INFO page for your NetLogo model and add the following
    (this example is for exercise 5.4)

\begin{verbatim}
# Exercise 5.4

When I ran the model ...
\end{verbatim}
  \item
    Write your answer in a Word file or other document file (text, PDF,
    etc.) and put the homework answers there. You can either make a
    separate document for each exercise (e.g., answers for exercise 5.4
    go in \texttt{ex\_5\_4.docx}, answers for exercise 5.7 go in
    \texttt{ex\_5\_7.docx}), or you can make a single document file
    (e.g., with the title \texttt{homework\_5\_answers.docx}) and put
    the homework answers in there, as follows:

    \begin{quote}
    \begin{itemize}
    \tightlist
    \item
      \textbf{Exercise 5.4:} When I changed the value of \emph{q},
      \ldots{}
    \item
      \textbf{Exercise 5.7:} If the butterflies stop when they reach a
      patch that has no higher neighbors \ldots{}
    \end{itemize}
    \end{quote}
  \item
    If your answers include diagrams or other kinds of figures or
    illustrations that are a pain to do in Word, you can hand-write your
    answers (\textbf{legibly!}), take a photo of them, and upload the
    photo.
  \end{enumerate}
\item
  \textbf{Graduate Students:} For exercise 5.8, you need to answer
  questions about your models. Do this in a file that you upload to
  Brightspace, or write your answers on paper and take a picture and
  upload it, or edit the ``Info'' page of your model and put the answer
  there.
\end{itemize}

\section{Wed., Sep.~17: Reproducing a model from its
ODD}\label{wed.-sep.-17-reproducing-a-model-from-its-odd}

\subsection{Homework}\label{homework-5}

\subsubsection{Homework Exercises:}\label{homework-exercises-5}

\begin{itemize}
\item
  \textbf{Graduate Students:}

  \begin{itemize}
  \tightlist
  \item
    Graduate students should do Railsback \& Grimm, Ex. 5.11
  \end{itemize}
\end{itemize}

\subsubsection{Notes on Homework:}\label{notes-on-homework-3}

\begin{itemize}
\item
  \textbf{Graduate Students:} You can download the journal article for
  this exercise,
  \href{/files/models/chapter_05/Jovani_Grimm_2008_Breeding.pdf}{R.
  Jovani \& V. Grimm. (2008) ``Breeding synchrony in colonial birds:
  From local stress to global harmony'', \emph{Proc. Royal Soc. London
  B} \textbf{275}, 1567--63} from the class web site,
  \url{https://ees4760.jgilligan.org/files/models/chapter_05/Jovani_Grimm_2008_Breeding.pdf}.

  You don't have to reproduce all of the figures in the paper. Focus on
  reproducing the information in Figure 2 (make a figure of colony
  synchrony versus NR and some other figures showing histograms of
  breeding dates for NR = 0.00, 0.08, 0.20, and 1.00 and comparing them
  to the histograms in Fig. 2.

  You may also (optionally) try to reproduce Fig. 1 and Fig. 4.

  Fig. 3 is very hard to reproduce because you would need to write a
  reporter to calculate the size of the colonies and that is quite
  difficult with what you know at this point about NetLogo programming,
  so I don't recommend this.

  \begin{itemize}
  \tightlist
  \item
    The paper by Jovani and Grimm forgot to specify the parameter
    \texttt{SD.} It should have the value 10.0.
  \end{itemize}
\end{itemize}

\section{Wed., Sep.~24: Testing and debugging
models}\label{wed.-sep.-24-testing-and-debugging-models}

\subsection{Homework}\label{homework-6}

\subsubsection{Homework Exercises:}\label{homework-exercises-6}

\begin{itemize}
\item
  \textbf{Everyone:}

  \begin{itemize}
  \tightlist
  \item
    Railsback \& Grimm, Ch. 6, Ex. 6.2, 6.3
  \end{itemize}
\item
  \textbf{Graduate Students,} also do the following:

  \begin{itemize}
  \tightlist
  \item
    Railsback \& Grimm, Ch. 6, Ex. 6.4, 6.5
  \end{itemize}
\end{itemize}

\subsubsection{Notes on Homework:}\label{notes-on-homework-4}

You do not need to turn anything in for Exercise 6.2. This is an
exercise to try for the experience

A hint for exercise 6.3: Patches have integer coordinates (representing
the center of the patch). How does the turtle determine the angle to
face during \texttt{setup}? How does it determine the angle to face in
\texttt{go-back}? Can you think of a different way to record the path so
the \texttt{go-back} exactly retraces the path it took during
\texttt{setup}? If you knew the direction the turtle was heading at each
step during \texttt{setup}, could you use this heading information to
exactly retrace the path?

\section{Mon., Sep.~29: Analyzing model
experiments}\label{mon.-sep.-29-analyzing-model-experiments}

\subsection{Homework}\label{homework-7}

\subsubsection{Homework Exercises:}\label{homework-exercises-7}

\begin{itemize}
\item
  \textbf{Everyone:}

  \begin{itemize}
  \tightlist
  \item
    Railsback \& Grimm, Ch. 8, Ex. 8.1, 8.2
  \item
    Railsback \& Grimm, Ch. 9, Ex. 9.3, 9.4
  \end{itemize}
\item
  \textbf{Graduate Students,} also do the following:

  \begin{itemize}
  \tightlist
  \item
    Railsback \& Grimm, Ch. 8, Ex. 8.3, 8.4
  \item
    Railsback \& Grimm, Ch. 9, Ex. 9.6
  \end{itemize}
\end{itemize}

\section{Wed., Oct.~1: Programming agent
sensing}\label{wed.-oct.-1-programming-agent-sensing}

\subsection{Homework}\label{homework-8}

\subsubsection{Homework Exercises:}\label{homework-exercises-8}

\textbf{Optional:} The following exercises are optional. You can turn
them in for extra credit.

\begin{itemize}
\tightlist
\item
  Railsback \& Grimm, Ch. 10, Ex. 10.1
\item
  Railsback \& Grimm, Ch. 11, Ex. 11.1
\end{itemize}

\section{Wed., Oct.~8: Team modeling project
presentations}\label{wed.-oct.-8-team-modeling-project-presentations}

\subsection{Homework}\label{homework-9}

\subsubsection{Homework Exercises:}\label{homework-exercises-9}

\begin{itemize}
\tightlist
\item
  Teams will give a presentation on their projects.
\end{itemize}

\subsubsection{Notes on Homework:}\label{notes-on-homework-5}

See the team project assignment sheet for details on what I expect for
the presentation.

\section{Wed., Oct.~15: Analysis of a published
model}\label{wed.-oct.-15-analysis-of-a-published-model}

\subsection{Homework}\label{homework-10}

\subsubsection{Homework Exercises:}\label{homework-exercises-10}

\begin{itemize}
\item
  Study the code and ODD of the model you chose for your semester
  research project, play with the model and run some BehaviorSpace
  experiments to examine its output.

  Turn in a 3--5 page (double-spaced) write-up about the model.
\end{itemize}

\subsubsection{Notes on Homework:}\label{notes-on-homework-6}

See the project assignment sheet for details about this assignment.

\section{Wed., Oct.~29: Research project
ODD}\label{wed.-oct.-29-research-project-odd}

\subsection{Homework}\label{homework-11}

\subsubsection{Homework Exercises:}\label{homework-exercises-11}

\begin{itemize}
\tightlist
\item
  Turn in an ODD for extending your chosen model to ask new questions.
\end{itemize}

\subsubsection{Notes on Homework:}\label{notes-on-homework-7}

See the semester research project assignment sheet for details.

\section{Wed., Nov.~12: Draft model code for research
project}\label{wed.-nov.-12-draft-model-code-for-research-project}

\subsection{Homework}\label{homework-12}

\subsubsection{Homework Exercises:}\label{homework-exercises-12}

\begin{itemize}
\item
  Turn in a draft \texttt{.nlogo} file with your modified model. The ODD
  for your modified model should be included in the ``Info'' section of
  the model.

  You should \emph{also} turn in a document that describes what you are
  satisfied with about your draft model and what problems you are
  struggling with.
\end{itemize}

\subsubsection{Notes on Homework:}\label{notes-on-homework-8}

The model code you turn in should run, but it does not need to be
perfect.

The point of this deadline is so that you can check in with me about how
things are going so I can give you feedback and suggestions.

See the project assignment sheet for details.

\section{Mon., Dec.~1: Research project
presentations}\label{mon.-dec.-1-research-project-presentations}

\subsection{Homework}\label{homework-13}

\subsubsection{Homework Exercises:}\label{homework-exercises-13}

\begin{itemize}
\tightlist
\item
  You will make a five-minute presentation in class about your model
  (You can use the full five minutes of talking and there will be an
  additional two minutes for questions).
\end{itemize}

\subsubsection{Notes on Homework:}\label{notes-on-homework-9}

There will not be time to go into all the details in your presentation,
so focus on:

\subsection{Structure:}\label{structure}

\begin{enumerate}
\tightlist
\item
  Start by describing your research question
\item
  Briefly describe your model
\item
  Describe the experiments you performed:

  \begin{itemize}
  \tightlist
  \item
    What outputs did you observe or measure?
  \item
    What parameters did you change?
  \item
    What did you see? (e.g., behaviorspace graphs or screenshots of the
    model)
  \end{itemize}
\item
  Conclusions
\end{enumerate}

\subsection{General recommendations:}\label{general-recommendations}

\begin{itemize}
\tightlist
\item
  5 slides, one minute each.
\item
  As much as you can, use figures (graphs, illustrations, images, etc.)
  instead of text.
\item
  Avoid slides with a lot of text
\end{itemize}

You can't say everything you did, so focus on the most interesting
highlights.

You can describe what you did in greater depth in your paper.

\subsection{Schedule of Presentations}\label{schedule-of-presentations}

\begin{longtable}[]{@{}cc@{}}
\toprule\noalign{}
Monday Dec.~2 & Wednesday Dec.~4 \\
\midrule\noalign{}
\endhead
\bottomrule\noalign{}
\endlastfoot
Ferna Alvarez-Carrascal & Savi Buluwana \\
Andrew Harder & Maya Maciel-Seidman \\
Xingyu Huang & Nadya Paschall \\
Erica Scarpitti & Rayshaun Pettit \\
Jeremy Tallon & Farzana Rahman \\
Jiaying Yang & Xiyu Wang \\
Samuel Yang & Hannah Wolf \\
\end{longtable}

\section{Wed., Dec.~3: Research project
presentations}\label{wed.-dec.-3-research-project-presentations}

\subsection{Homework}\label{homework-14}

\subsubsection{Homework Exercises:}\label{homework-exercises-14}

\begin{itemize}
\tightlist
\item
  You will make a five-minute presentation in class about your model
  (You can use the full five minutes of talking and there will be an
  additional two minutes for questions).
\end{itemize}

\subsubsection{Notes on Homework:}\label{notes-on-homework-10}

There will not be time to go into all the details in your presentation,
so focus on:

\subsection{Structure:}\label{structure-1}

\begin{enumerate}
\tightlist
\item
  Start by describing your research question
\item
  Briefly describe your model
\item
  Describe the experiments you performed:

  \begin{itemize}
  \tightlist
  \item
    What outputs did you observe or measure?
  \item
    What parameters did you change?
  \item
    What did you see? (e.g., behaviorspace graphs or screenshots of the
    model)
  \end{itemize}
\item
  Conclusions
\end{enumerate}

\subsection{General recommendations:}\label{general-recommendations-1}

\begin{itemize}
\tightlist
\item
  5 slides, one minute each.
\item
  As much as you can, use figures (graphs, illustrations, images, etc.)
  instead of text.
\item
  Avoid slides with a lot of text
\end{itemize}

You can't say everything you did, so focus on the most interesting
highlights.

You can describe what you did in greater depth in your paper.

\subsection{Schedule of
Presentations}\label{schedule-of-presentations-1}

\begin{longtable}[]{@{}cc@{}}
\toprule\noalign{}
Monday Dec.~2 & Wednesday Dec.~4 \\
\midrule\noalign{}
\endhead
\bottomrule\noalign{}
\endlastfoot
Ferna Alvarez-Carrascal & Savi Buluwana \\
Andrew Harder & Maya Maciel-Seidman \\
Xingyu Huang & Nadya Paschall \\
Erica Scarpitti & Rayshaun Pettit \\
Jeremy Tallon & Farzana Rahman \\
Jiaying Yang & Xiyu Wang \\
Samuel Yang & Hannah Wolf \\
\end{longtable}

\end{document}
