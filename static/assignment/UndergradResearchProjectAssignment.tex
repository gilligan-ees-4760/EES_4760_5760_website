\documentclass{jghandout}
\usepackage{listings}
\lstset{basicstyle=\ttfamily,breaklines=true}

\LoadCourseInfo
\iffalse
\newif{\ifisbn}
\isbnfalse

\newcommand{\shortdept}{EES}
\newcommand{\longdept}{EES}
\newcommand{\coursenum}{4760/5760}
\newcommand{\sectionnum}{}
\newcommand{\shortcoursetitle}{Agent- and Individual-Based Computational Modeling}
\newcommand{\longcoursetitle}{\shortcoursetitle}
\newcommand{\semester}{Fall}
\newcommand{\yeartaught}{2021}

\newcommand{\Classroom}{Stevenson 2220}
\newcommand{\ClassMeetings}{TR 11:10--12:25\ \Classroom}

\newcommand{\ProfName}{Jonathan Gilligan}
\newcommand{\ShortProfName}{J.\ Gilligan}
\newcommand{\ProfTitle}{Prof.\ Gilligan}
\newcommand{\ProfDescr}{Associate Professor of Earth \& Environmental Sciences\\
Associate Professor of Civil \& Environmental Engineering}
\newcommand{\ProfOffice}{Office: Stevenson 5735 (Stevenson \#5, 7\textsuperscript{th} floor)}
\newcommand{\ProfPhone}{Phone: 322-2420}
\newcommand{\ProfEmail}{\href{mailto:jonathan.gilligan@vanderbilt.edu}%
{\nolinkurl{jonathan.gilligan@vanderbilt.edu}}}
\newcommand{\ProfWeb}{\href{https://www.jonathangilligan.org}%
{\nolinkurl{www.jonathangilligan.org}}}

\newcommand{\ProfOfficeHours}{Office Hours:
	Tues. 4:00--5:00 pm,
	Wed.  7:00--8:00 pm (on Zoom),
	or by appointment.%
	} % TODO: Office Hours

\TAmalefalse
\newcommand{\TaName}{Katerine ``Kat'' Turk}
\newcommand{\TaTitle}{Ms.\ Turk}
\newcommand{\TaOffice}{Stevenson 5743}
\newcommand{\TaOfficeLoc}{Office: \TaOffice}
\newcommand{\TaOfficeHours}{Office Hours:
Tues.\ 10:00--11:00 am,
Wed.\ 11:00 am--12:00 pm,
or by appointment.%
}
\TaPhonefalse
\ifTaPhone
\newcommand{\TaPhone}{\qqq}
\fi
\newcommand{\TaEmail}{\href{mailto:katherine.a.turk@vanderbilt.edu}%
{\nolinkurl{katherine.a.turk@vanderbilt.edu}}}
%{\href{qqq}%
%{\nolinkurl{qqq}}}

\newcommand{\FinalExamDay}{Thursday}   %TODO: Final Exam Date & Time
\newcommand{\ShortFinalExamDay}{Thurs.{}}   %TODO: Final Exam Date & Time
\newcommand{\FinalExamMonth}{December} %TODO: Final Exam Date & Time
\newcommand{\ShortFinalExamMonth}{Dec.{}} %TODO: Final Exam Date & Time
\newcommand{\FinalExamDate}{16}  %TODO: Final Exam Date & Time
\newcommand{\FinalExamTime}{3:00--5:00~pm}  %TODO: Final Exam Date & Time
\newcommand{\FinalExamStartTime}{3:00~pm}
\newcommand{\FinalExamEndTime}{5:00~pm}
\newcommand{\FinalExamRoom}{\Classroom\ (our regular classroom)}  %TODO: Final Exam Date & Time

\newif\ifAltFinal
\AltFinaltrue

\ifAltFinal
\newcommand{\AltFinalExamDay}{Monday}   %TODO: Alternate Final Exam Date & Time
\newcommand{\ShortAltFinalExamDay}{Mon.{}}   %TODO: Alternate Final Exam Date & Time
\newcommand{\AltFinalExamMonth}{December} %TODO: Alternate Final Exam Date & Time
\newcommand{\ShortAltFinalExamMonth}{Dec.{}} %TODO: Final Exam Date & Time
\newcommand{\AltFinalExamDate}{13}  %TODO: Alternate Final Exam Date & Time
\newcommand{\AltFinalExamTime}{12:00--2:00~pm}  %TODO: Alternate Final Exam Date & Time
\newcommand{\AltFinalExamStartTime}{12:00~pm}
\newcommand{\AltFinalExamEndTime}{2:00~pm}
\newcommand{\AltFinalExamRoom}{\FinalExamRoom}  %TODO: Alternate Final Exam Date & Time
\fi

\newcommand{\NetLogo}{{\sffamily\scshape NetLogo}}
\newcommand{\NetLogoURL}{\url{http://ccl.northwestern.edu/netlogo/}}

\iffalse
\newcommand{\Epstein}{Growing Artificial Societies}
\newcommand{\ShortEpstein}{Growing}
\newcommand{\MedEpstein}{\Epstein}
\newcommand{\LongEpstein}{Joshua M. Epstein \& Robert Axtell,
	\emph{Growing Artificial Societies: Social Science from the Bottom Up\/}
	(Brookings/MIT, 1996\ifisbn ; ISBN 978-0-262-55025-3\fi )%
	}

\newcommand{\Growing}{\Epstein}
\newcommand{\ShortGrowing}{\ShortEpstein}
\newcommand{\MedGrowing}{\MedEpstein}
\newcommand{\LongGrowing}{\LongEpstein}
\fi

\newcommand{\Railsback}{Agent-Based and Individual-Based Modeling}
\newcommand{\ShortRailsback}{Agent-Based Modeling}
\newcommand{\MedRailsback}{\Railsback}
\newcommand{\AltShortRailsback}{Railsback \& Grimm}
\newcommand{\ShorttestRailsback}{AB\&IBM}
\newcommand{\LongRailsback}{Steven F. Railsback \& Volker Grimm,
    \emph{Agent-Based and Individual-Based Modeling: A Practical Introduction\/}
    (Princeton University Press, 2012\ifisbn ; ISBN 978-0-691-13674-5\fi )%
}
\newcommand{\RailsbackURL}{\url{http://www.railsback-grimm-abm-book.com/index.html}}


\newcommand{\AbmBook}{\Railsback}
\newcommand{\ShortAbmBook}{\ShortRailsback}
\newcommand{\MedAbmBook}{\MedRailsback}
\newcommand{\LongAbmBook}{\LongRailsback}


\fi

\DueDate{}
\LongDueDate{Proposal due Sept.~23.\\
    Model analysis due Oct.~7.\\
    ODD for extending the model due Oct.~25.\\
    Draft model code due Nov.~8.\\
    Presentations Dec.~2--4.\\
    Written Report due Dec.~6}

\SetTitle{Undergraduate Student Research Project}

\setcounter{secnumdepth}{0}

\SetupHandout{}

\date{\longduedate}
\fancyhead[R]{}
\fancyhead[L]{}
\fancyhead[C]{\ShortTitle}


\begin{document}
    \maketitle

\noindent
You will choose a research area, find an existing agent-based model relevant to
that area, study the model, develop a new research question and modify the model
to address it, and perform and analyze experiments with your modified model.

The project is broken into the following steps:
\begin{description}
  \item[Project Proposal:] \textbf{By the end of the day Monday, Sept.~23, turn
   in a one-to-two page (double-spaced) proposal describing the topic you want
   to work on and identifying a published open-source model you want to work
   with and how you think you might want to extend it.}

    You should consult the textbook, the Model Library in NetLogo, the reading
    and computational tools resource I handed out on the first day of class
    (also on Brightspace), and the repositories of models listed in that
    handout.

    The supplementary textbook, \emph{Modeling Social Behavior}, by
    Paul Smaldino, also includes a number of interesting models that you could
    adapt for research projects.

    \item[Model Analysis:] Study the ODD and the code of the model (if the model
     does not have a published ODD, try to write your own to describe what it
     does). Play with the model and run some BehaviorSpace experiments to examine
     its output.
    \textbf{By the end of the day Oct.~7 (before midnight), turn in a 3--5 page
     (double-spaced) write-up about the model}:
    \begin{itemize}
      \item Describe the purpose of the model, the important features about its
       design and structure.
      \item Discuss whether there are weaknesses in the model design and whether
       you might have represented the system differently if you were writing a
       model from scratch.
      \item Present the output from the model (Behaviorspace or interactively
      playing with it) and discuss what you learn about the system from that.
      \item Discuss something that you don't learn from the  model that you
      would like to study by extending the model.
    \end{itemize}
    \item[Extending the Model:] By Oct.~25, turn in an ODD for extending the model
    to ask new questions and begin to edit the model to implement your new ODD.

    \textbf{By the end of the day Fri.\ Nov. 8, turn in a draft model that
    runs, together with a description of what you are satisfied with and what
    problems you are still struggling with.}
    It is not necessary that the model be perfect. This due date is to get you
    to check in with me so I can give you feedback on how things are going and
    make suggestions.

    \item[Running Model Experiments:] Once your model is running and passes your
    tests, you should use it to explore the research questions in your proposal.
    You should think about the topics we read about in sections III and IV
    (``Pattern-Oriented Modeling'' and ``Model Analysis'') in
    \emph{\MedRailsback}.

    \item[Analyzing and Interpreting Results:] What do you learn from your
    model? Graphing output or performing statistical analysis with Excel or
    other tools, consider what you learned. Pay special attention to patterns
    and emergence. Did the interesting phenomena you observed only occur for
    specific values of parameters, or did they occur for broad ranges of those
    parameters?
    \item[Presentation and Report:] You will make a ten-minute presentation
    (seven minutes of talking and three minutes for questions) about your
    model. There will not be time to go into details, so focus on the big
    question you were addressing, a short description of the approach you
    took to writing a model to address that question, and what you learned
    from running the model.

    By the end of the day Friday December 6, turn in a written report about
    your research project. Your report should follow the model of a research
    Report in \emph{Science\/} magazine:
    \begin{description}
        \item[Title:] Maximum of 90 characters
        \item[Author:] Your full name. If this is a team project, the full
        names of all the authors.
        \item[Abstract] A one-paragraph abstract that explains to a general
        reader why you did this project and why the results were important.
        \item[Text] Up to 2000 words (around eight pages, double-spaced), not
        including the abstract, references, notes, or captions. Up to four
        figures or tables.
      \item[Contributions:] If there is more than one author, there should be
        a short section at the end of the paper, just before the references,
        which clearly states the contributions each author made. If all
        authors contributed equally, you may say so. If the work was divided,
       a statement such as the following should explain each author's
       contributions:
       \begin{quote}
         \textbf{Author contributions:} Both authors designed the research
         together.
         E.K.B. wrote the ODD, and programmed the model.
         J.J.N. ran the model experiments, conducted the statistical analysis
         of the results, and prepared the figures.
         Both authors discussed the results and wrote the manuscript together.
       \end{quote}
        \item[References:] Citations
        should use an author-date style, and
        references can be in any standard bibliographic format.
        \item[Supplementary Materials] The detailed description of your model
        (ODD and \NetLogo\ code) should not be part of the paper. You should
        submit these as a separate file or group of files called ``Supplementary
        Materials.'' (The ODD can be either a separate document or you can put it
        in the ``Info'' tab of your model).
        \setlength{\parindent}{1em}

        Your supplementary materials can also include additional figures, tables,
        or text that will not fit in the paper.

        The main text (the paper) should stand alone in describing the
        problem, the general approach you took with your model, what you
        found, and why it's significant.

        The supplementary materials should include any details that someone would
        need in order to test and reproduce your results, but which are not
        crucial for a non-expert to understand the big picture. The supplementary
        materials should also include your model, any data files you used as
        input for running your model, and any computer code or spreadsheets
        you used to analyze your model output. Supplementary materials can also
        include additional text describing details that did not fit in the
        main paper.
    \end{description}

    \sloppy
    \begin{itemize}
      \item The main paper should be in a Word or PDF document with the filename
        \lstinline|<lastname>_<firstname>_final_paper.docx| or
        \lstinline|<lastname>_<firstname>_final_paper.pdf| (with your own name in
        place of \lstinline|<lastname>| and \lstinline|<firstname>|).
      \item Make a ZIP file called \lstinline|supporting_material.zip|,
        containing your supporting materials (ODD, \NetLogo\ model, etc.) and upload
        your paper along with the supporting materials to Brightspace.

        There are many free software apps for creating and working with ZIP files,
        including 7Zip on Windows and Keka on MacOS.
    \end{itemize}
\end{description}
\end{document}

\section{Example Topics}
Some possible topics include:
\begin{itemize}
    \item Does adding lanes to a freeway relieve traffic congestion?
    \item How would electric vehicles affect the stability of the electricity
     grid.
    \item What kinds of community interactions lead to sustainable vs.\
     unsustainable management of shared resources (water supply, fishing grounds,
     pasture, etc.)?
    \item How do different kinds of insurance and disaster relief policies affect
    vulnerability to natural hazards.
    \item How does habitat segmentation affect biodiversity and wildlife
    conservation.
    \item How does pesticide use affect the vulnerability of forests to insect
    infestations?
    \item How does ``monoculture'' agriculture affect the vulnerability of crops
    to pests?
    \item Can social networks play an important role in promoting household energy
    or water conservation?
    \item How do individual decisions not to vaccinate children against infectious
    diseases affect risks of epidemics?
    \item What conditions are necessary for economic markets to operate
    efficiently?
    \item Could charging micropayments for sending email solve the problem of
    SPAM?
    \item How do economic markets, peer pressure, and government regulation
    compare as strategies for managing pollution.
    \item Managing congestion in computer networks.
    \item Identifying optimal strategies in game theory (advanced topic)
\end{itemize}

\end{document}

