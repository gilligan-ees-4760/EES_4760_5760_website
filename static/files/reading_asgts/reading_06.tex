% Options for packages loaded elsewhere
\PassOptionsToPackage{unicode}{hyperref}
\PassOptionsToPackage{hyphens}{url}
%
\documentclass[
]{article}
\usepackage{amsmath,amssymb}
\usepackage{iftex}
\ifPDFTeX
  \usepackage[T1]{fontenc}
  \usepackage[utf8]{inputenc}
  \usepackage{textcomp} % provide euro and other symbols
\else % if luatex or xetex
  \usepackage{unicode-math} % this also loads fontspec
  \defaultfontfeatures{Scale=MatchLowercase}
  \defaultfontfeatures[\rmfamily]{Ligatures=TeX,Scale=1}
\fi
\usepackage{lmodern}
\ifPDFTeX\else
  % xetex/luatex font selection
\fi
% Use upquote if available, for straight quotes in verbatim environments
\IfFileExists{upquote.sty}{\usepackage{upquote}}{}
\IfFileExists{microtype.sty}{% use microtype if available
  \usepackage[]{microtype}
  \UseMicrotypeSet[protrusion]{basicmath} % disable protrusion for tt fonts
}{}
\makeatletter
\@ifundefined{KOMAClassName}{% if non-KOMA class
  \IfFileExists{parskip.sty}{%
    \usepackage{parskip}
  }{% else
    \setlength{\parindent}{0pt}
    \setlength{\parskip}{6pt plus 2pt minus 1pt}}
}{% if KOMA class
  \KOMAoptions{parskip=half}}
\makeatother
\usepackage{xcolor}
\usepackage[margin=1in]{geometry}
\usepackage{graphicx}
\makeatletter
\newsavebox\pandoc@box
\newcommand*\pandocbounded[1]{% scales image to fit in text height/width
  \sbox\pandoc@box{#1}%
  \Gscale@div\@tempa{\textheight}{\dimexpr\ht\pandoc@box+\dp\pandoc@box\relax}%
  \Gscale@div\@tempb{\linewidth}{\wd\pandoc@box}%
  \ifdim\@tempb\p@<\@tempa\p@\let\@tempa\@tempb\fi% select the smaller of both
  \ifdim\@tempa\p@<\p@\scalebox{\@tempa}{\usebox\pandoc@box}%
  \else\usebox{\pandoc@box}%
  \fi%
}
% Set default figure placement to htbp
\def\fps@figure{htbp}
\makeatother
\setlength{\emergencystretch}{3em} % prevent overfull lines
\providecommand{\tightlist}{%
  \setlength{\itemsep}{0pt}\setlength{\parskip}{0pt}}
\setcounter{secnumdepth}{-\maxdimen} % remove section numbering
\usepackage{mathptmx}
\usepackage{float}
\usepackage{booktabs}
\usepackage[version=4]{mhchem}
% \usepackage{longtable}
\usepackage{caption}
\usepackage{array}
\usepackage{multirow}
\usepackage{wrapfig}
\usepackage{float}
\usepackage{colortbl}
% \usepackage{pdflscape}
\usepackage{tabu}
\usepackage{threeparttable}
\usepackage{threeparttablex}
% \usepackage[normalem]{ulem}
\usepackage{makecell}
\usepackage{xcolor}
\usepackage{makecmds}

\newcommand{\COO}{\ce{CO2}}
\newcommand{\methane}{\ce{CH4}}
\newcommand{\degC}{^\circ \mathrm{C}}
\newcommand{\degF}{^\circ \mathrm{F}}
\newcommand{\water}{\mathrm{H_2O}}
\newcommand{\carb}{\ce{CO3^2-}}
\newcommand{\bicarb}{\ce{HCO3-}}
\newcommand{\carbonic}{\ce{H2CO3}}
\newcommand{\Hplus}{\ce{H+}}
\newcommand{\OH}{\ce{OH-}}
\newcommand{\silica}{\ce{SiO2}}
\newcommand{\calcite}{\ce{CaCO3}}
\newcommand{\Caplus}{\ce{Ca^2+}}
\newcommand{\silicate}{\ce{SiO3^2-}}
\newcommand{\CaSi}{\ce{CaSiO3}}
\newcommand{\pH}{p\ce{H}}
\newcommand{\permil}{\permille}

\newcommand{\NetLogo}{{\sffamily\scshape NetLogo}}
\newcommand{\NetLogoURL}{\url{https://ccl.northwestern.edu/netlogo/}}

\newcommand{\alintertext}[1]{%
  \noalign{%
    \vskip\belowdisplayshortskip
    \vtop{\hsize=\linewidth#1\par
      \expandafter}%
    \expandafter\prevdepth\the\prevdepth
  }%
}

\makeenvironment{Shaded}{\begin{snugshade}\small}{\end{snugshade}}
\makecommand{\AttributeTok}[1]{\textcolor[rgb]{0.42,0.37,0.12}{#1}}
\usepackage{bookmark}
\IfFileExists{xurl.sty}{\usepackage{xurl}}{} % add URL line breaks if available
\urlstyle{same}
\hypersetup{
  pdftitle={Using models for science},
  hidelinks,
  pdfcreator={LaTeX via pandoc}}

\title{Using models for science}
\author{}
\date{\vspace{-2.5em}Reading for Class \#6: Monday, Sep 08, 2025}

\begin{document}
\maketitle

{
\setcounter{tocdepth}{2}
\tableofcontents
}
\subsection{Reading:}\label{reading}

\subsubsection{Required Reading
(everyone):}\label{required-reading-everyone}

\begin{itemize}
\tightlist
\item
  Agent-Based and Individual-Based Modeling, Ch. 5.
\end{itemize}

\subsubsection{Optional Extra Reading:}\label{optional-extra-reading}

\begin{itemize}
\tightlist
\item
  Modeling Social Behavior, Ch. 3.
\end{itemize}

\subsubsection{Reading Notes:}\label{reading-notes}

This reading sets the stage for answering the big question, ``How can we
use agent-based models to do science?'' There are several aspects to
this question, which this chapter will introduce:

\begin{enumerate}
\def\labelenumi{\arabic{enumi}.}
\tightlist
\item
  How can we produce quantitative output from our models?
\item
  How can your models read and write data to and from files? (This is
  important for connecting your model to other parts of your project)
\item
  How should we test our models to make sure they do what we think they
  do? (More on this in Chapter 6)
\item
  Making your research reproducible by using version control and
  documentation.
\end{enumerate}

A number of you may like to use Excel or statistical analysis tools,
such as \texttt{R}, \texttt{SPSS}, or \texttt{Stata}. The material in
this chapter about importing and exporting data using text or
\texttt{.csv} files will be very useful for this. By default, NetLogo
only allows you to read in data in simple text files. However, if comes
with some extensions that you can use to read in data from other common
file formats, including \texttt{.csv} and ArcGIS shapefiles and raster
(grid) files.

If you want to read in data from csv files, you may want to look at the
documentation for the \texttt{csv} extension to NetLogo. To use it, you
just put the line \texttt{includes\ {[}csv{]}} as the first line of your
model, and then use functions from the extension, such as
\texttt{let\ data\ csv:from-file\ "myfile.csv"}.

To read in date from ArcGIS files, look at the documentation for the
\texttt{GIS} extension. You would put the line
\texttt{extensions\ {[}\ gis\ {]}} as the first line of your model, and
then use functions, such as \texttt{gis:load-dataset}, which can load
vector shape files (\texttt{.shp}) and raster grid files (\texttt{.grd}
or \texttt{.asc}). The GIS extension offers a lot of functions for
working with vector and raster GIS data. If you're interested in using
GIS data in your models, take a look at the GIS examples in the NetLogo
model library.

You can download the
\href{/files/models/chapter_05/ElevationData.txt}{data file} with the
elevations for the realistic butterfly model from
\url{https://ees4760.jgilligan.org/files/models/chapter_05/ElevationData.txt}

The chapter from \emph{Modeling Social Behavior} is an optional
supplementary reading. I strongly recommend that graduate students and
students interested in social systems and social science read this.

The main reading for today, from Railsback \& Grimm presents a model of
an ecological system. This chapter presents a famous agent-based model
of racial segregation in housing. This model is historically important,
and also controversial. It was perhaps the first agent-based model ever
used to study a research problem in social sicence, and it was written
by a researcher who went on to win the Nobel Prize in economics.
However, the model is problematic and has been criticized because it is
often interpreted in ways that minimize the role of institutional racism
in driving segregation (e.g., government policies that explicitly
prohibited racially integrated housing across large parts of the entire
United States). This chapter presents the model and also discusses the
challenges of using it effectively for science and the importance of
considering the actual historical context of the social system being
modeled. Specifically, this model does not account for the historical
segregationist policies, so it's important not to assume that
experiments using this model can tell us about real-world segregation in
the U.S., but nonetheless, the model can help us identicy potentially
important obstacles to remedying the segregated housing patterns that
those policies produced.

\end{document}
