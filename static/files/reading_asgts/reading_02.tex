% Options for packages loaded elsewhere
\PassOptionsToPackage{unicode}{hyperref}
\PassOptionsToPackage{hyphens}{url}
%
\documentclass[
]{article}
\usepackage{amsmath,amssymb}
\usepackage{iftex}
\ifPDFTeX
  \usepackage[T1]{fontenc}
  \usepackage[utf8]{inputenc}
  \usepackage{textcomp} % provide euro and other symbols
\else % if luatex or xetex
  \usepackage{unicode-math} % this also loads fontspec
  \defaultfontfeatures{Scale=MatchLowercase}
  \defaultfontfeatures[\rmfamily]{Ligatures=TeX,Scale=1}
\fi
\usepackage{lmodern}
\ifPDFTeX\else
  % xetex/luatex font selection
\fi
% Use upquote if available, for straight quotes in verbatim environments
\IfFileExists{upquote.sty}{\usepackage{upquote}}{}
\IfFileExists{microtype.sty}{% use microtype if available
  \usepackage[]{microtype}
  \UseMicrotypeSet[protrusion]{basicmath} % disable protrusion for tt fonts
}{}
\makeatletter
\@ifundefined{KOMAClassName}{% if non-KOMA class
  \IfFileExists{parskip.sty}{%
    \usepackage{parskip}
  }{% else
    \setlength{\parindent}{0pt}
    \setlength{\parskip}{6pt plus 2pt minus 1pt}}
}{% if KOMA class
  \KOMAoptions{parskip=half}}
\makeatother
\usepackage{xcolor}
\usepackage[margin=1in]{geometry}
\usepackage{graphicx}
\makeatletter
\newsavebox\pandoc@box
\newcommand*\pandocbounded[1]{% scales image to fit in text height/width
  \sbox\pandoc@box{#1}%
  \Gscale@div\@tempa{\textheight}{\dimexpr\ht\pandoc@box+\dp\pandoc@box\relax}%
  \Gscale@div\@tempb{\linewidth}{\wd\pandoc@box}%
  \ifdim\@tempb\p@<\@tempa\p@\let\@tempa\@tempb\fi% select the smaller of both
  \ifdim\@tempa\p@<\p@\scalebox{\@tempa}{\usebox\pandoc@box}%
  \else\usebox{\pandoc@box}%
  \fi%
}
% Set default figure placement to htbp
\def\fps@figure{htbp}
\makeatother
\setlength{\emergencystretch}{3em} % prevent overfull lines
\providecommand{\tightlist}{%
  \setlength{\itemsep}{0pt}\setlength{\parskip}{0pt}}
\setcounter{secnumdepth}{-\maxdimen} % remove section numbering
\usepackage{mathptmx}
\usepackage{float}
\usepackage{booktabs}
\usepackage[version=4]{mhchem}
% \usepackage{longtable}
\usepackage{caption}
\usepackage{array}
\usepackage{multirow}
\usepackage{wrapfig}
\usepackage{float}
\usepackage{colortbl}
% \usepackage{pdflscape}
\usepackage{tabu}
\usepackage{threeparttable}
\usepackage{threeparttablex}
% \usepackage[normalem]{ulem}
\usepackage{makecell}
\usepackage{xcolor}
\usepackage{makecmds}

\newcommand{\COO}{\ce{CO2}}
\newcommand{\methane}{\ce{CH4}}
\newcommand{\degC}{^\circ \mathrm{C}}
\newcommand{\degF}{^\circ \mathrm{F}}
\newcommand{\water}{\mathrm{H_2O}}
\newcommand{\carb}{\ce{CO3^2-}}
\newcommand{\bicarb}{\ce{HCO3-}}
\newcommand{\carbonic}{\ce{H2CO3}}
\newcommand{\Hplus}{\ce{H+}}
\newcommand{\OH}{\ce{OH-}}
\newcommand{\silica}{\ce{SiO2}}
\newcommand{\calcite}{\ce{CaCO3}}
\newcommand{\Caplus}{\ce{Ca^2+}}
\newcommand{\silicate}{\ce{SiO3^2-}}
\newcommand{\CaSi}{\ce{CaSiO3}}
\newcommand{\pH}{p\ce{H}}
\newcommand{\permil}{\permille}

\newcommand{\NetLogo}{{\sffamily\scshape NetLogo}}
\newcommand{\NetLogoURL}{\url{https://ccl.northwestern.edu/netlogo/}}

\newcommand{\alintertext}[1]{%
  \noalign{%
    \vskip\belowdisplayshortskip
    \vtop{\hsize=\linewidth#1\par
      \expandafter}%
    \expandafter\prevdepth\the\prevdepth
  }%
}

\makeenvironment{Shaded}{\begin{snugshade}\small}{\end{snugshade}}
\makecommand{\AttributeTok}[1]{\textcolor[rgb]{0.42,0.37,0.12}{#1}}
\usepackage{bookmark}
\IfFileExists{xurl.sty}{\usepackage{xurl}}{} % add URL line breaks if available
\urlstyle{same}
\hypersetup{
  pdftitle={The computer modeling cycle},
  hidelinks,
  pdfcreator={LaTeX via pandoc}}

\title{The computer modeling cycle}
\author{}
\date{\vspace{-2.5em}Reading for Class \#2: Monday, Aug 25, 2025}

\begin{document}
\maketitle

{
\setcounter{tocdepth}{2}
\tableofcontents
}
\subsection{Reading:}\label{reading}

\subsubsection{Required Reading
(everyone):}\label{required-reading-everyone}

\begin{itemize}
\tightlist
\item
  Agent-Based and Individual-Based Modeling, Ch. 1.
\item
  Handout: \href{/files/reading/Tyson_Artificial_Societies_1997.pdf}{P.
  Tyson, ``Artificial Societies,'' Technology Review \textbf{100} (3),
  15--17 (1997).}.
\end{itemize}

\subsubsection{Optional Extra Reading:}\label{optional-extra-reading}

\begin{itemize}
\tightlist
\item
  Modeling Social Behavior, Ch. 1.
\end{itemize}

\subsubsection{Reading Notes:}\label{reading-notes}

This reading sets the stage for answering the questions:

\begin{enumerate}
\def\labelenumi{\arabic{enumi}.}
\tightlist
\item
  What is computational modeling and why is it useful in social and
  natural science research?
\item
  What are agent based models? How are they different from other kinds
  of models? What makes them useful for scientific research?
\end{enumerate}

The reading introduces the idea of a \textbf{modeling cycle}. You should
understand the different steps in the modeling cycle. You should also
think about why Railsback and Grimm describe modeling as a cycle, as
opposed to a linear process with a start and stop.

As to what makes agent-based modeling special, Steven Railsback and
Volker Grimm are ecologists and \emph{Agent-Based Modeling} emphasizes
aspects of agent-based modeling that are well suited for studying
ecological systems. Others, such as social scientists, emphasize the
aspects of agent-based modeling that are well suited for problems in
social science. And still others, such as computer scientists, emphasize
aspects of automated and autonomous things (ranging from packets of data
on a network to swarms of robots or flying drones that need to
coordinate their activities and avoid collisions). What all of these
approaches have in common are their use of individuals or
\textbf{agents} (what is an agent?), which inhabit some kind of space or
\textbf{environment} (this could be physical space or an abstract space,
such as a computer network). Agents \textbf{interact} with each other
and with the environment, and they make \textbf{decisions} according to
rules.

The article ``Artificial Life'' gives you a feel for how an early
agent-based model called ``Sugarscape'' was used as part of a very
influential research project in the 1990s. Joshua Epstein and Robert
Axtell who wrote Sugarscape are highly respected pioneers in agent-based
modeling and the Sugarscape model set off a revolution in agent-based
modeling by showing that a very simple model could reproduce complex
phenomena that are observed in real societies. As you read through this
article, think about what the different applications of agent-based
models have in common. Do these suggest questions that you might be
interested in exploring with agent-based models. Do you have questions,
as you read this, about whether computer modeling can really tell you
about real societies?

The chapter from \emph{Modeling Social Behavior} is optional, but
strongly recommended for graduate students and for students interested
in applications of agent-based modeling to social systems and social
science.

Agent-based models are often used to examine \textbf{emergent}
phenomena. Neither reading describes clearly what \emph{emergence}
means. There is no simple definition, but during the semester we will
pay a lot of attention to learning about emergence and trying to
understand it. Do not worry if you don't understand emergence at this
point. Emergence is difficult to put into words, and it's much easier to
understand from experience. Over the course of the semester, we will
work together to understand what emergence is and how to study it.

\end{document}
