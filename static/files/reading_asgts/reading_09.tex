% Options for packages loaded elsewhere
\PassOptionsToPackage{unicode}{hyperref}
\PassOptionsToPackage{hyphens}{url}
%
\documentclass[
]{article}
\usepackage{amsmath,amssymb}
\usepackage{iftex}
\ifPDFTeX
  \usepackage[T1]{fontenc}
  \usepackage[utf8]{inputenc}
  \usepackage{textcomp} % provide euro and other symbols
\else % if luatex or xetex
  \usepackage{unicode-math} % this also loads fontspec
  \defaultfontfeatures{Scale=MatchLowercase}
  \defaultfontfeatures[\rmfamily]{Ligatures=TeX,Scale=1}
\fi
\usepackage{lmodern}
\ifPDFTeX\else
  % xetex/luatex font selection
\fi
% Use upquote if available, for straight quotes in verbatim environments
\IfFileExists{upquote.sty}{\usepackage{upquote}}{}
\IfFileExists{microtype.sty}{% use microtype if available
  \usepackage[]{microtype}
  \UseMicrotypeSet[protrusion]{basicmath} % disable protrusion for tt fonts
}{}
\makeatletter
\@ifundefined{KOMAClassName}{% if non-KOMA class
  \IfFileExists{parskip.sty}{%
    \usepackage{parskip}
  }{% else
    \setlength{\parindent}{0pt}
    \setlength{\parskip}{6pt plus 2pt minus 1pt}}
}{% if KOMA class
  \KOMAoptions{parskip=half}}
\makeatother
\usepackage{xcolor}
\usepackage[margin=1in]{geometry}
\usepackage{graphicx}
\makeatletter
\newsavebox\pandoc@box
\newcommand*\pandocbounded[1]{% scales image to fit in text height/width
  \sbox\pandoc@box{#1}%
  \Gscale@div\@tempa{\textheight}{\dimexpr\ht\pandoc@box+\dp\pandoc@box\relax}%
  \Gscale@div\@tempb{\linewidth}{\wd\pandoc@box}%
  \ifdim\@tempb\p@<\@tempa\p@\let\@tempa\@tempb\fi% select the smaller of both
  \ifdim\@tempa\p@<\p@\scalebox{\@tempa}{\usebox\pandoc@box}%
  \else\usebox{\pandoc@box}%
  \fi%
}
% Set default figure placement to htbp
\def\fps@figure{htbp}
\makeatother
\setlength{\emergencystretch}{3em} % prevent overfull lines
\providecommand{\tightlist}{%
  \setlength{\itemsep}{0pt}\setlength{\parskip}{0pt}}
\setcounter{secnumdepth}{-\maxdimen} % remove section numbering
\usepackage{mathptmx}
\usepackage{float}
\usepackage{booktabs}
\usepackage[version=4]{mhchem}
% \usepackage{longtable}
\usepackage{caption}
\usepackage{array}
\usepackage{multirow}
\usepackage{wrapfig}
\usepackage{float}
\usepackage{colortbl}
% \usepackage{pdflscape}
\usepackage{tabu}
\usepackage{threeparttable}
\usepackage{threeparttablex}
% \usepackage[normalem]{ulem}
\usepackage{makecell}
\usepackage{xcolor}
\usepackage{makecmds}

\newcommand{\COO}{\ce{CO2}}
\newcommand{\methane}{\ce{CH4}}
\newcommand{\degC}{^\circ \mathrm{C}}
\newcommand{\degF}{^\circ \mathrm{F}}
\newcommand{\water}{\mathrm{H_2O}}
\newcommand{\carb}{\ce{CO3^2-}}
\newcommand{\bicarb}{\ce{HCO3-}}
\newcommand{\carbonic}{\ce{H2CO3}}
\newcommand{\Hplus}{\ce{H+}}
\newcommand{\OH}{\ce{OH-}}
\newcommand{\silica}{\ce{SiO2}}
\newcommand{\calcite}{\ce{CaCO3}}
\newcommand{\Caplus}{\ce{Ca^2+}}
\newcommand{\silicate}{\ce{SiO3^2-}}
\newcommand{\CaSi}{\ce{CaSiO3}}
\newcommand{\pH}{p\ce{H}}
\newcommand{\permil}{\permille}

\newcommand{\NetLogo}{{\sffamily\scshape NetLogo}}
\newcommand{\NetLogoURL}{\url{https://ccl.northwestern.edu/netlogo/}}

\newcommand{\alintertext}[1]{%
  \noalign{%
    \vskip\belowdisplayshortskip
    \vtop{\hsize=\linewidth#1\par
      \expandafter}%
    \expandafter\prevdepth\the\prevdepth
  }%
}

\makeenvironment{Shaded}{\begin{snugshade}\small}{\end{snugshade}}
\makecommand{\AttributeTok}[1]{\textcolor[rgb]{0.42,0.37,0.12}{#1}}
\usepackage{bookmark}
\IfFileExists{xurl.sty}{\usepackage{xurl}}{} % add URL line breaks if available
\urlstyle{same}
\hypersetup{
  pdftitle={Emergence},
  hidelinks,
  pdfcreator={LaTeX via pandoc}}

\title{Emergence}
\author{}
\date{\vspace{-2.5em}Reading for Class \#9: Wednesday, Sep 17, 2025}

\begin{document}
\maketitle

{
\setcounter{tocdepth}{2}
\tableofcontents
}
\subsection{Reading:}\label{reading}

\subsubsection{Required Reading
(everyone):}\label{required-reading-everyone}

\begin{itemize}
\tightlist
\item
  Agent-Based and Individual-Based Modeling, Ch. 8.
\end{itemize}

\subsubsection{Reading Notes:}\label{reading-notes}

This is a major chapter. Emergence is one of the most important concepts
in agent-based modeling, so pay close attention to the discussion in
this chapter and think about how you can measure and assess emergence.

This chapter also introduces a very important tool for doing experiments
in NetLogo: \textbf{BehaviorSpace}. BehaviorSpace lets you repeatedly
run a NetLogo model while varying the settings of any of the controls on
your user interface. Where there is randomness (stochasticity) in the
model, you can perform many runs at each set of control settings. This
will let us perform \textbf{sensitivity analysis} to determine whether a
certain emergent phenomenon we are investigating happens only for values
of the parameters within a narrow range, or whether it happens over a
wide range of the parameters. It will let us determine which parameters
are most important for the phenomenon.

For homework and your modeling projects you will use BehaviorSpace
extensively. BehaviorSpace outputs large amounts of data to
\texttt{.csv} files, which you can read into Excel, R, SPSS, or another
tool where you can do statistical analysis and generate plots such as
the ones in figures 8.3, 8.5, and 8.6.

The format in which BehaviorSpace saves its data is very annoying to
deal with in many tools. Indeed, it's almost impossible to do anything
useful with it in Excel. Because of this, I have written a tool called
\texttt{analyzeBehaviorspace} that can read the output of a
BehaviorSpace run and allow you to interactively graph it and
re-organize the data to make it more useful.

You can either use this tool online in a web browser at
\url{https://analyze-behaviorspace.jgilligan.org} or install it on your
own computer. For details, see the description of
\texttt{analyzeBehaviorspace} on the ``Reading Resources and Computing
Tools for Research'' handout.

As you read the chapter, be sure to try out the experiments with the
birth-and-death model and the flocking model. Try reading the output of
the behaviorspace runs into analyzeBehaviorspace (the web version or a
local version installed on your computer), or your favorite statistical
analysis software and try to generate plots similar to figures 8.3, 8.5,
and 8.6.

If you have time, try to play around with BehaviorSpace using those
models (varying different parameters) or other models from the NetLogo
library to explore the ways that changing parameters affects the models'
behavior.

\end{document}
