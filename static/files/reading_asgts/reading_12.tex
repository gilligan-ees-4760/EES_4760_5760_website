% Options for packages loaded elsewhere
\PassOptionsToPackage{unicode}{hyperref}
\PassOptionsToPackage{hyphens}{url}
%
\documentclass[
]{article}
\usepackage{amsmath,amssymb}
\usepackage{iftex}
\ifPDFTeX
  \usepackage[T1]{fontenc}
  \usepackage[utf8]{inputenc}
  \usepackage{textcomp} % provide euro and other symbols
\else % if luatex or xetex
  \usepackage{unicode-math} % this also loads fontspec
  \defaultfontfeatures{Scale=MatchLowercase}
  \defaultfontfeatures[\rmfamily]{Ligatures=TeX,Scale=1}
\fi
\usepackage{lmodern}
\ifPDFTeX\else
  % xetex/luatex font selection
\fi
% Use upquote if available, for straight quotes in verbatim environments
\IfFileExists{upquote.sty}{\usepackage{upquote}}{}
\IfFileExists{microtype.sty}{% use microtype if available
  \usepackage[]{microtype}
  \UseMicrotypeSet[protrusion]{basicmath} % disable protrusion for tt fonts
}{}
\makeatletter
\@ifundefined{KOMAClassName}{% if non-KOMA class
  \IfFileExists{parskip.sty}{%
    \usepackage{parskip}
  }{% else
    \setlength{\parindent}{0pt}
    \setlength{\parskip}{6pt plus 2pt minus 1pt}}
}{% if KOMA class
  \KOMAoptions{parskip=half}}
\makeatother
\usepackage{xcolor}
\usepackage[margin=1in]{geometry}
\usepackage{graphicx}
\makeatletter
\newsavebox\pandoc@box
\newcommand*\pandocbounded[1]{% scales image to fit in text height/width
  \sbox\pandoc@box{#1}%
  \Gscale@div\@tempa{\textheight}{\dimexpr\ht\pandoc@box+\dp\pandoc@box\relax}%
  \Gscale@div\@tempb{\linewidth}{\wd\pandoc@box}%
  \ifdim\@tempb\p@<\@tempa\p@\let\@tempa\@tempb\fi% select the smaller of both
  \ifdim\@tempa\p@<\p@\scalebox{\@tempa}{\usebox\pandoc@box}%
  \else\usebox{\pandoc@box}%
  \fi%
}
% Set default figure placement to htbp
\def\fps@figure{htbp}
\makeatother
\setlength{\emergencystretch}{3em} % prevent overfull lines
\providecommand{\tightlist}{%
  \setlength{\itemsep}{0pt}\setlength{\parskip}{0pt}}
\setcounter{secnumdepth}{-\maxdimen} % remove section numbering
\usepackage{mathptmx}
\usepackage{float}
\usepackage{booktabs}
\usepackage[version=4]{mhchem}
% \usepackage{longtable}
\usepackage{caption}
\usepackage{array}
\usepackage{multirow}
\usepackage{wrapfig}
\usepackage{float}
\usepackage{colortbl}
% \usepackage{pdflscape}
\usepackage{tabu}
\usepackage{threeparttable}
\usepackage{threeparttablex}
% \usepackage[normalem]{ulem}
\usepackage{makecell}
\usepackage{xcolor}
\usepackage{makecmds}

\newcommand{\COO}{\ce{CO2}}
\newcommand{\methane}{\ce{CH4}}
\newcommand{\degC}{^\circ \mathrm{C}}
\newcommand{\degF}{^\circ \mathrm{F}}
\newcommand{\water}{\mathrm{H_2O}}
\newcommand{\carb}{\ce{CO3^2-}}
\newcommand{\bicarb}{\ce{HCO3-}}
\newcommand{\carbonic}{\ce{H2CO3}}
\newcommand{\Hplus}{\ce{H+}}
\newcommand{\OH}{\ce{OH-}}
\newcommand{\silica}{\ce{SiO2}}
\newcommand{\calcite}{\ce{CaCO3}}
\newcommand{\Caplus}{\ce{Ca^2+}}
\newcommand{\silicate}{\ce{SiO3^2-}}
\newcommand{\CaSi}{\ce{CaSiO3}}
\newcommand{\pH}{p\ce{H}}
\newcommand{\permil}{\permille}

\newcommand{\NetLogo}{{\sffamily\scshape NetLogo}}
\newcommand{\NetLogoURL}{\url{https://ccl.northwestern.edu/netlogo/}}

\newcommand{\alintertext}[1]{%
  \noalign{%
    \vskip\belowdisplayshortskip
    \vtop{\hsize=\linewidth#1\par
      \expandafter}%
    \expandafter\prevdepth\the\prevdepth
  }%
}

\makeenvironment{Shaded}{\begin{snugshade}\small}{\end{snugshade}}
\makecommand{\AttributeTok}[1]{\textcolor[rgb]{0.42,0.37,0.12}{#1}}
\usepackage{bookmark}
\IfFileExists{xurl.sty}{\usepackage{xurl}}{} % add URL line breaks if available
\urlstyle{same}
\hypersetup{
  pdftitle={Adaptive Behavior and Objectives},
  hidelinks,
  pdfcreator={LaTeX via pandoc}}

\title{Adaptive Behavior and Objectives}
\author{}
\date{\vspace{-2.5em}Reading for Class \#12: Monday, Sep 29, 2025}

\begin{document}
\maketitle

{
\setcounter{tocdepth}{2}
\tableofcontents
}
\subsection{Reading:}\label{reading}

\subsubsection{Required Reading
(everyone):}\label{required-reading-everyone}

\begin{itemize}
\tightlist
\item
  Agent-Based and Individual-Based Modeling, Ch. 11.
\end{itemize}

\subsubsection{Reading Notes:}\label{reading-notes}

Agents' behavior often consists of trying to achieve some
\textbf{objective}.

I have discussed the way that Adam Smith's ``invisible hand of the
market'' is a kind of agent-based view of a nation's economy: Each
person (agent) has an objective of trying to maximize his or her own
wealth (that's the agent's \textbf{micromotive}), and in doing so, the
population of agents manages unintentionally to maximize the total
wealth of the nation (an emergent \textbf{macrobehavior} that results
from the collective interactions of the agents and their micromotives).

For Darwin, agents whose objectives are to survive and reproduce under
changing environmental conditions achieve emergent phenomena of
evolution and speciation.

If we are going to program an agent-based model to simulate such an
economy (we saw this in the Sugarscape models), you need to program your
agents to try to achieve their objective (maximize their wealth). There
are two approaches to this:

\begin{enumerate}
\def\labelenumi{\arabic{enumi}.}
\tightlist
\item
  You could program a sophisticated strategy into your agents.
\item
  You could program a simple strategy into your agents, but give them
  the ability to learn from their experience and adapt their behavior
  according to what they learn. (see section 11.3 for details and an
  example)
\end{enumerate}

This chapter discusses different kinds of objectives you might have your
agents employ. An important concept from decision theory and behavioral
economics that might be new to you is \textbf{satisficing}. This term,
introduced by Herbert Simon\footnote{Herbert Simon (1916--2001) was a
  fascinating intellectual. Kind of a renaissance scholar, he made major
  contributions to political science, economics, cognitive psychology,
  and artificial intelligence. He won the Nobel Prize for economics in
  1978. His publications have been cited more than 250,000 times and
  even 24 years after his death, they are still cited more than 10,000
  times per year.} in 1956, refers to making decisions by choosing a
``good-enough'' option when it would take too much time and effort to
determine which option is the absolute best. See section 11.4 for
details and an example.

\end{document}
