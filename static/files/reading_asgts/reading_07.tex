% Options for packages loaded elsewhere
\PassOptionsToPackage{unicode}{hyperref}
\PassOptionsToPackage{hyphens}{url}
%
\documentclass[
]{article}
\usepackage{amsmath,amssymb}
\usepackage{iftex}
\ifPDFTeX
  \usepackage[T1]{fontenc}
  \usepackage[utf8]{inputenc}
  \usepackage{textcomp} % provide euro and other symbols
\else % if luatex or xetex
  \usepackage{unicode-math} % this also loads fontspec
  \defaultfontfeatures{Scale=MatchLowercase}
  \defaultfontfeatures[\rmfamily]{Ligatures=TeX,Scale=1}
\fi
\usepackage{lmodern}
\ifPDFTeX\else
  % xetex/luatex font selection
\fi
% Use upquote if available, for straight quotes in verbatim environments
\IfFileExists{upquote.sty}{\usepackage{upquote}}{}
\IfFileExists{microtype.sty}{% use microtype if available
  \usepackage[]{microtype}
  \UseMicrotypeSet[protrusion]{basicmath} % disable protrusion for tt fonts
}{}
\makeatletter
\@ifundefined{KOMAClassName}{% if non-KOMA class
  \IfFileExists{parskip.sty}{%
    \usepackage{parskip}
  }{% else
    \setlength{\parindent}{0pt}
    \setlength{\parskip}{6pt plus 2pt minus 1pt}}
}{% if KOMA class
  \KOMAoptions{parskip=half}}
\makeatother
\usepackage{xcolor}
\usepackage[margin=1in]{geometry}
\usepackage{graphicx}
\makeatletter
\newsavebox\pandoc@box
\newcommand*\pandocbounded[1]{% scales image to fit in text height/width
  \sbox\pandoc@box{#1}%
  \Gscale@div\@tempa{\textheight}{\dimexpr\ht\pandoc@box+\dp\pandoc@box\relax}%
  \Gscale@div\@tempb{\linewidth}{\wd\pandoc@box}%
  \ifdim\@tempb\p@<\@tempa\p@\let\@tempa\@tempb\fi% select the smaller of both
  \ifdim\@tempa\p@<\p@\scalebox{\@tempa}{\usebox\pandoc@box}%
  \else\usebox{\pandoc@box}%
  \fi%
}
% Set default figure placement to htbp
\def\fps@figure{htbp}
\makeatother
\setlength{\emergencystretch}{3em} % prevent overfull lines
\providecommand{\tightlist}{%
  \setlength{\itemsep}{0pt}\setlength{\parskip}{0pt}}
\setcounter{secnumdepth}{-\maxdimen} % remove section numbering
\usepackage{mathptmx}
\usepackage{float}
\usepackage{booktabs}
\usepackage[version=4]{mhchem}
% \usepackage{longtable}
\usepackage{caption}
\usepackage{array}
\usepackage{multirow}
\usepackage{wrapfig}
\usepackage{float}
\usepackage{colortbl}
% \usepackage{pdflscape}
\usepackage{tabu}
\usepackage{threeparttable}
\usepackage{threeparttablex}
% \usepackage[normalem]{ulem}
\usepackage{makecell}
\usepackage{xcolor}
\usepackage{makecmds}

\newcommand{\COO}{\ce{CO2}}
\newcommand{\methane}{\ce{CH4}}
\newcommand{\degC}{^\circ \mathrm{C}}
\newcommand{\degF}{^\circ \mathrm{F}}
\newcommand{\water}{\mathrm{H_2O}}
\newcommand{\carb}{\ce{CO3^2-}}
\newcommand{\bicarb}{\ce{HCO3-}}
\newcommand{\carbonic}{\ce{H2CO3}}
\newcommand{\Hplus}{\ce{H+}}
\newcommand{\OH}{\ce{OH-}}
\newcommand{\silica}{\ce{SiO2}}
\newcommand{\calcite}{\ce{CaCO3}}
\newcommand{\Caplus}{\ce{Ca^2+}}
\newcommand{\silicate}{\ce{SiO3^2-}}
\newcommand{\CaSi}{\ce{CaSiO3}}
\newcommand{\pH}{p\ce{H}}
\newcommand{\permil}{\permille}

\newcommand{\NetLogo}{{\sffamily\scshape NetLogo}}
\newcommand{\NetLogoURL}{\url{https://ccl.northwestern.edu/netlogo/}}

\newcommand{\alintertext}[1]{%
  \noalign{%
    \vskip\belowdisplayshortskip
    \vtop{\hsize=\linewidth#1\par
      \expandafter}%
    \expandafter\prevdepth\the\prevdepth
  }%
}

\makeenvironment{Shaded}{\begin{snugshade}\small}{\end{snugshade}}
\makecommand{\AttributeTok}[1]{\textcolor[rgb]{0.42,0.37,0.12}{#1}}
\usepackage{bookmark}
\IfFileExists{xurl.sty}{\usepackage{xurl}}{} % add URL line breaks if available
\urlstyle{same}
\hypersetup{
  pdftitle={Testing and validating models},
  hidelinks,
  pdfcreator={LaTeX via pandoc}}

\title{Testing and validating models}
\author{}
\date{\vspace{-2.5em}Reading for Class \#7: Wednesday, Sep 10, 2025}

\begin{document}
\maketitle

{
\setcounter{tocdepth}{2}
\tableofcontents
}
\subsection{Reading:}\label{reading}

\subsubsection{Required Reading
(everyone):}\label{required-reading-everyone}

\begin{itemize}
\tightlist
\item
  Agent-Based and Individual-Based Modeling, Ch. 6.
\end{itemize}

\subsubsection{Reading Notes:}\label{reading-notes}

No one writes perfect programs. Errors in programs controlling medical
equipment have killed people. Errors in computer models and data
analysis code have not had such dire results, but have wasted lots of
time for researchers and have caused public policy to proceed on
incorrect assumptions. In many cases, these errors were uncovered only
after a great deal of frustration because the original researchers would
not share their computer codes with others who were suspicious of their
results.

You can never be certain that your model is correct, but the more
aggressively you check for errors the more confident you can be that it
does not have major problems.

Two very important things you can do to ensure that your research does
not suffer a similar fate are:

\begin{enumerate}
\def\labelenumi{\arabic{enumi}.}
\item
  Test your code. Assume your program has errors in it and make the
  search for those errors a priority in your programming process. Some
  things you can do in this regard are:

  \begin{itemize}
  \tightlist
  \item
    Write your code with tests that will help you find errors.
  \item
    Work with a partner: after one of you writes code, the other should
    read it and check for errors.
  \item
    Break your program up into small chunks. It is easier to test and
    find bugs if you are looking at a short block of code than if you
    are looking at hundreds of lines of code.
  \item
    Independently reimplement submodels and check whether they agree
    with the submodel you are using.
  \end{itemize}
\item
  Publish your code. If you trust your results and believe they are
  important enough to publish in a book or journal, then you should make
  your code available (there are many free sites, such as
  \href{https://github.com}{\texttt{github.com}} and
  \href{https://openabm.org}{\texttt{openabm.org}} where people can
  publish their models and other computer code).

  The more that other researchers can read your code, the greater the
  probability that they will find any errors, and if you make it easy
  for others to use your code, it will help science because other people
  can build on your work, and it will help your reputation because when
  other people use your model or other code they are likely to cite the
  publication in which you first announced it, so your work will get
  attention.

  Many scholarly journals demand that you make your code available as a
  condition for publishing your paper, and federal funding agencies are
  increasingly requiring that any research funded by their grants must
  make its code and data available to other researchers and the public.
\end{enumerate}

\textbf{The Cultural Dissemination Model}

The
\href{//files/models/chapter_06/axelrod_culture_dissemination_1997.pdf}{paper
describing the culture dissemination model} and a
\href{/files/models/chapter_06/CultureDissemination_Untested.nlogo}{NetLogo
model} that implements the ODD, but with many errors, can be downloaded
from the class web site:

\begin{itemize}
\tightlist
\item
  \url{https://ees4760.jgilligan.org/files/models/chapter_06/axelrod_culture_dissemination_1997.pdf},
\item
  and
  \url{https://ees4760.jgilligan.org/files/models/chapter_06/CultureDissemination_Untested.nlogo}.
\end{itemize}

\end{document}
