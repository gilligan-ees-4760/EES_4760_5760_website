\documentclass[]{article}
\usepackage{lmodern}
\usepackage{amssymb,amsmath}
\usepackage{ifxetex,ifluatex}
\usepackage{fixltx2e} % provides \textsubscript
\ifnum 0\ifxetex 1\fi\ifluatex 1\fi=0 % if pdftex
  \usepackage[T1]{fontenc}
  \usepackage[utf8]{inputenc}
\else % if luatex or xelatex
  \ifxetex
    \usepackage{mathspec}
  \else
    \usepackage{fontspec}
  \fi
  \defaultfontfeatures{Ligatures=TeX,Scale=MatchLowercase}
\fi
% use upquote if available, for straight quotes in verbatim environments
\IfFileExists{upquote.sty}{\usepackage{upquote}}{}
% use microtype if available
\IfFileExists{microtype.sty}{%
\usepackage{microtype}
\UseMicrotypeSet[protrusion]{basicmath} % disable protrusion for tt fonts
}{}
\usepackage[margin=1in]{geometry}
\usepackage{hyperref}
\hypersetup{unicode=true,
            pdfborder={0 0 0},
            breaklinks=true}
\urlstyle{same}  % don't use monospace font for urls
\usepackage{graphicx,grffile}
\makeatletter
\def\maxwidth{\ifdim\Gin@nat@width>\linewidth\linewidth\else\Gin@nat@width\fi}
\def\maxheight{\ifdim\Gin@nat@height>\textheight\textheight\else\Gin@nat@height\fi}
\makeatother
% Scale images if necessary, so that they will not overflow the page
% margins by default, and it is still possible to overwrite the defaults
% using explicit options in \includegraphics[width, height, ...]{}
\setkeys{Gin}{width=\maxwidth,height=\maxheight,keepaspectratio}
\IfFileExists{parskip.sty}{%
\usepackage{parskip}
}{% else
\setlength{\parindent}{0pt}
\setlength{\parskip}{6pt plus 2pt minus 1pt}
}
\setlength{\emergencystretch}{3em}  % prevent overfull lines
\providecommand{\tightlist}{%
  \setlength{\itemsep}{0pt}\setlength{\parskip}{0pt}}
\setcounter{secnumdepth}{5}
% Redefines (sub)paragraphs to behave more like sections
\ifx\paragraph\undefined\else
\let\oldparagraph\paragraph
\renewcommand{\paragraph}[1]{\oldparagraph{#1}\mbox{}}
\fi
\ifx\subparagraph\undefined\else
\let\oldsubparagraph\subparagraph
\renewcommand{\subparagraph}[1]{\oldsubparagraph{#1}\mbox{}}
\fi

%%% Use protect on footnotes to avoid problems with footnotes in titles
\let\rmarkdownfootnote\footnote%
\def\footnote{\protect\rmarkdownfootnote}

%%% Change title format to be more compact
\usepackage{titling}

% Create subtitle command for use in maketitle
\providecommand{\subtitle}[1]{
  \posttitle{
    \begin{center}\large#1\end{center}
    }
}

\setlength{\droptitle}{-2em}

  \title{}
    \pretitle{\vspace{\droptitle}}
  \posttitle{}
    \author{}
    \preauthor{}\postauthor{}
    \date{}
    \predate{}\postdate{}


\begin{document}

\hypertarget{the-business-investor-model}{%
\section{THE BUSINESS INVESTOR
MODEL}\label{the-business-investor-model}}

This model was produced by S. Railsback and V. Grimm for Chapter 10 of
the book Agent-Based and Individual-Based Modeling, 2nd edition (2019).

This is the first version of the model, as described in Section 10.4.1.

\hypertarget{overview}{%
\subsection{OVERVIEW}\label{overview}}

\hypertarget{purpose-and-patterns}{%
\subsubsection{PURPOSE AND PATTERNS}\label{purpose-and-patterns}}

The primary purpose of this model is to explore the effects of
sensing---what information agents have and how they obtain it---on
emergent outcomes of a model in which agents make adaptive decisions
using sensed information. The model uses investment decisions as an
example, but it is not intended to represent any real investment
approach or business sector. (In fact, you will see by the end of
chapter 12 that models like this one that are designed mainly to explore
the system-level effects of sensing and other concepts can produce very
different results depending on exactly how those concepts are
implemented. What can we learn from such models if their results depend
on what seem like details? In part III of this course we will learn to
solve this problem by tying models more closely to real systems.)

This model can be thought of as approximately representing people who
buy and operate local businesses: it assumes investors are familiar with
businesses they could buy within a limited range of their own
experience. The investment ``environment'' assumes that there is no cost
of entering or switching businesses (e.g., as if capital to buy a
business is borrowed and the repayment is included in the annual profit
calculation), that high profits are rarer than low profits, and that
risk of failure is unrelated to profit. We can use the model to address
questions such as how the average wealth of the investors, and how
evenly wealth is distributed among individuals, depends on how much
information the investors can sense.

Because this model is conceptual and not intended to represent a
specific real system, we can only use general patterns as criteria for
its usefulness. These patterns are that investor decisions depend on the
profit and risk ``landscape,'' and that the decisions change with
investor wealth.

\hypertarget{entities-state-variables-and-scales}{%
\subsubsection{ENTITIES, STATE VARIABLES, AND
SCALES}\label{entities-state-variables-and-scales}}

The entities in this model are investor agents (turtles) and business
alternatives (patches) that vary in profit and risk. The investors have
state variables for their location in the space and for their current
wealth ($W$, in money units).

The landscape is a grid of business patches, each of which has two
static variables: the annual net profit that a business there would
provide (\(P\), in money units such as dollars per year) and the
annual risk of that business failing and its investor losing all its
wealth (\(F\), as probability per year). This landscape is 19 × 19
patches in size with no wrapping at its edges.

The model time step is 1 year, and simulations run for 50 years.

\hypertarget{process-overview-and-scheduling}{%
\subsubsection{PROCESS OVERVIEW AND
SCHEDULING}\label{process-overview-and-scheduling}}

The model includes the following actions that are executed in this order
each time step:

\emph{Investor repositioning.} The investors decide whether any similar
business (adjacent patch) offers a better tradeoff of profit and risk;
if so, they ``reposition'' and transfer their investment to that patch,
by moving there. The repositioning submodel is described at element 7.
Only one investor can occupy a patch at a time. The agents execute this
repositioning action in randomized order.

\emph{Accounting.} The investors update their wealth state variable.
\(W\) is set equal to the previous wealth plus the profit of the
agent's current patch. However, unexpected failure of the business is
also included in the accounting action. This event is a stochastic
function of \(F\) at the investor's patch. If a uniform random number
between zero and one is less than \(F\), then the business fails: the
investor's \(W\) is set to zero, but the investor stays in the model
and continues to behave in the same way.

\emph{Output.} The View, plots, and an output file are updated.

\hypertarget{design-concepts}{%
\subsection{DESIGN CONCEPTS}\label{design-concepts}}

\emph{Basic principles.} The basic topic of this model is how agents
make decisions involving tradeoffs between several objectives---here,
increasing profit and decreasing risk.

\emph{Emergence.} The model's primary output is mean investor wealth
over time. Important secondary outputs are the mean profit and risk
chosen by investors over time, and the number of investors who have
suffered a failure. These outputs emerge from how individual investors
make their tradeoff between profit and risk but also from the ``business
climate'': the ranges of \(P\) and \(F\) values among patches and
the number of investors competing for locations on the landscape.

\emph{Adaptive behavior.} The adaptive behavior of investor agents is
repositioning: the decision of which neighboring business to move to (or
whether to stay put), considering the profit and risk of these
alternatives. Each time step, investors can reposition to any unoccupied
one of their adjacent patches or retain their current position. In this
version of the model, investors use a simplified microeconomic analysis
to make their decision, moving to the patch providing the highest value
of an objective function.

\emph{Objective.} (In economics, the term utility is used for the
objective that agents seek.) Investors rate business alternatives by a
utility measure that represents their expected future wealth at the end
of a time horizon (\(T\), a number of future years; we use 5) if they
buy and operate the business. This expected future wealth is a function
of the investor's current wealth and the profit and failure risk offered
by the patch:

$$
U = (W + TP) (1 - F)^T
$$

where \(U\) is the expected utility for the patch, \(W\) is the
investor's current wealth, and \(P\) and \(F\) are defined above.
The term $(W + TP)$ estimates investor wealth at the end of
the time horizon if no failures occur. The term \((1 - F)^T\)
is the probability of not having a failure over the time horizon; it
reduces utility more as failure risk increases. (Economists might expect
to use a utility measure such as present value that includes a discount
rate to reduce the value of future profit. We ignore discounting to keep
this model simple.)

\emph{Prediction.} The utility measure estimates utility over a time
horizon by using the explicit prediction that \(P\) and \(F\) will
remain constant over the time horizon. This assumption is accurate here
because the patches' \(P\) and \(F\) values are static.

\emph{Sensing.} The investor agents are assumed to know the profit and
risk at their own patch and the adjacent neighbor patches, without
error.

\emph{Interaction.} The investors interact with each other only
indirectly via competition for patches; an investor cannot take over a
business (move into a patch) that is already occupied by another
investor. Investors execute their repositioning action in randomized
order, so there is no hierarchy in this competition: investors with
higher wealth have no advantage over others in competing for locations.

\emph{Stochasticity.} The initial state of the model is stochastic: the
values of \(P\) and \(F\) of each patch, and initial investor
locations, are set randomly. \emph{Stochasticity} is thus used to
simulate an investment environment where alternatives are highly
variable and risk is not correlated with profit. The values of \(P\)
are drawn from an exponential distribution, which produces many patches
with low profits and a few patches with high profits (we will learn more
about this and other random number distributions in chapter 15). The
random exponential distribution of \(P\) makes the results of this
model especially variable, even among model runs with the same parameter
values. Whether each investor fails each year is also stochastic, a
simple way to represent risk. The investor reposition action uses
stochasticity only in the very unlikely event that more than one
potential destination patch offers the same highest utility; when there
is such a tie, the agent randomly chooses one of the tied patches to
move to.

\emph{Observation.} The View shows the location of each agent on the
investment landscape. Having the investors put their pen down lets us
observe how many patches each has used. Graphs show the mean profit and
risk experienced by investors, and mean investor wealth over time. The
standard deviation in wealth among investors is also graphed as a simple
and appropriate measure of how evenly wealth is distributed among
investors.

\emph{Learning} and \emph{collectives} are not represented.

\hypertarget{details}{%
\subsection{DETAILS}\label{details}}

\hypertarget{initialization}{%
\subsubsection{INITIALIZATION}\label{initialization}}

The value of \(P\) for each patch is drawn from a random exponential
distribution with a mean of 5000. The value of \(F\) for each patch
is drawn randomly from a uniform real number distribution with a minimum
of 0.01 and a maximum of 0.1.

Twenty-five investor agents are initialized and put in random patches,
but investors cannot be placed in a patch already occupied by another
investor. Their wealth state variable \(W\) is initialized to zero.

\hypertarget{input-data}{%
\subsubsection{INPUT DATA}\label{input-data}}

No time-series inputs are used.

\hypertarget{submodels}{%
\subsubsection{SUBMODELS}\label{submodels}}

\emph{Investor repositioning.} An investor identifies all the businesses
that it could invest in: any of the neighboring eight (or fewer if on
the edge of the space) patches that are unoccupied, plus its current
patch. The investor then determines which of these alternatives provides
the highest value of the utility function, and moves (or stays) there.

\emph{Accounting.} This action is fully described above (``Process
overview and scheduling'').


\end{document}
